\newpage
\section{Conclusion}
\label{sec:conlusion}

\subsection{Travaux réalisés}
\label{sub:travauxrealises}
A partir des travaux de \nicolas, j'ai pu réaliser différentes améliorations, chronologiquement :
\begin{itemize}
%\item Franchissement d'obstacles
\item Prise en compte d'obstacles 3D complexes
\item Utilisation de PQP pour la détection de collisions
\item Connexion de trajectoires
\item Contrôle du robot en temps réel
\item Utilisation de la \emph{motion capture} pour le déplacement des obstacles
\item Correction du \emph{drift} en temps réel
\item Génération des \emph{swept volumes} pour le corps entier
\item Amélioration de la vitesse du RRT
\end{itemize}
%\newpage

\subsection{Travaux restants}
\label{sub:travauxrestant}

Il reste cependant quelques modifications à apporter :
\begin{itemize}
\item Prise en compte de la dynamique du robot (ZMP multi-body)
%\item Amélioration des trajectoires obtenues par le RRT
\item Acquisition de l'environnement via la vision pour rendre le robot complètement autonome
%\item Création des volumes balayés plus légers pour l'intégralité du corps du robot
\item Utilisation d'un \emph{buffer circulaire} comme zone tampon pour le contrôle
%\item Privilégier la marche avant sans trop contraindre la recherche RRT
\end{itemize}
%\newpage

\subsection{Conclusion générale}
\label{sub:conclusion}

Ce stage m'a permis de découvrir le monde de la robotique humanoïde qui est très différent de la robotique industrielle que l'on a pu apercevoir à l'IFMA. 

%J'ai appris à bien utiliser les outils disponibles sous linux (\emph{ssh}, \emph{scp}, \emph{make}, \emph{cmake}, librairies dynamiques, ...) 
\`A travers la lecture des articles, j'ai appris les notions de stabilité, de recherche de trajectoires, que j'ai pu ensuite appliquer sur un robot réel. 

Avant mon stage, plusieurs algorithmes de replanification existaient dans la littérature \cite{Chestnutt:ICRA:2005}, mais ceux-ci utilisent seulement des obstacles 2D (ou la projection des obstacles 3D sur le sol) ce qui ne permet pas d'utiliser toutes les capacités d'un robot humanoïde, c'est-à-dire l'enjambement. D'autres algorithmes \cite{Kuffner03} prennent en compte les obstacles 3D mais n'utilisent qu'un faible nombre de pas. 
J'ai donc utilisé un algorithme de planification utilisant des obstacles 3D \cite{baudouin:humanoids:11} afin de développer de la replanification en temps réel.
Mes principales contributions ont été l'optimisation du code-source pour planification plus rapide, ainsi que l'utilisation de fonctions de connexion de trajectoires.

Ayant travaillé avec Linux pendant six mois j'ai appris à mieux utiliser les fonctionnalités telles que \emph{ssh}, \emph{scp}, \emph{make}, \emph{cmake}, \emph{gnuplot} et l'utilité de créer nos propre librairies dynamiques lors du travail collaboratif. D'autre part, l'utilisation permanente de \emph{git} au sein du LAAS et du JRL a permis de travailler à plusieurs simultanément sur les mêmes codes-sources et de transmettre plus facilement mon travail.
