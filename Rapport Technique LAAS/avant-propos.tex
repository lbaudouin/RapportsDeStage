\section*{Avant Propos}
\addcontentsline{toc}{section}{\numberline{}Avant-propos}

Ayant toujours été attiré par la robotique, j'ai décidé de choisir ce thème pour mon stage IFMA\footnote{Institut Français de Mécanique Avancée} en entreprise qui correspondait également à mon stage de Master recherche, effectué en double cursus.
Après avoir discuté avec \olivier{} et \eiichi nous avons défini un sujet : la replanification en temp-réel pour les robots humano\"\i des.

J'ai travaillé au Japon, plus précisement au JRL\footnote{Joint Robotics Laboratory, c.f. \ref{sub:JRL}} à Tsukuba. Après le tremblement de terre et l'incident nucléaire j'ai décidé, avec l'accord de mes tuteurs, de terminer mon stage en France au LAAS\footnote{Laboratoire d'Analyse et d'Architecture des Systèmes, c.f. \ref{sub:LAAS}} au sein duquel travaille \florent avec qui j'avais eu l'occasion de correspondre via des réunions Skype.
Le changement de laboratoire n'a pas modifié le sujet du stage mais a apporté quelques changements au niveau de la réalisation des expériences comme vous pourrez le voir dans la partie \ref{sub:env}.
