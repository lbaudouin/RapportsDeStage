\documentclass{article}
\usepackage{CJK}

\usepackage[overlap,CJK]{ruby}    % Furigana support
\renewcommand{\rubysize}{0.5}     % Furigana size
\renewcommand{\rubysep}{-0.3ex}   % Spacing between Furigana and Kanji

\date{}

\begin{document}
\begin{CJK*}{UTF8}{min}

  \title{\ruby{第}{だい} 33 \ruby{課}{か}\\
    La forme imp\'erative}
  \maketitle

  \section{\ruby{文}{ぶん}\ruby{型}{けい}}

  \begin{itemize}
  \item いそげ。\\
    D\'ep\^echons!
  \item さわるな。\\
    Ne touches pas!
  \item \ruby{立入禁止}{たちいりきんし} は \ruby{入}{はい}るなと
    いう \ruby{意味}{いみ} です。\\
    ``Interdit d'entrer'' signifie qu'il est interdit d'enter.

  \item ミラーさん は \ruby{来週}{らいしゅう} \ruby{大阪}{おさか}へ
    \ruby{出張}{しゅっちょう}すると \ruby{言}{い}って いました。\\
    Mira a dit qu'elle partait en voyage d'affaire \`a
    Osaka la semaine prochaine.
  \end{itemize}

  ``V forme dictionnaire + な'' permet de former l'imp\'eratif n\'egatif.

  「と いう」 se traduit par ``qui signifie'' ou ``qui s'appelle''.


  \section{\ruby{例}{れい}\ruby{文}{ぶん}}

  \begin{itemize}
  \item だめた゛。もう \ruby{走れない}{はしれない}。\\
    ・・・\underline{\ruby{頑張}{がんば}れ}。あと 1000メートルだ。\\
    Impossible. Je ne peux plus courir.\\
    \ldots Courage! Plus que 1000 m\`etres.

  \item もう \ruby{時間}{じかん} が ない。\\
    ・・・まだ 1\ruby{分}{ぶん} ある。\underline{あきらめるな}。ファイト!\\
    On n'a plus le temps.\\
    Il reste encore une minute. N'abandonnes pas. Fight!

  \item あそこ に \ruby{何}{なん}と \ruby{書}{か}いて あるんですか。\\
    ・・・「\underline{\ruby{止}{と}まれ}」と \ruby{書}{か}いて あります。\\
    Qu'y-a-t-il d'\'ecrit ici?\\
    Il est \'ecrit ``arr\^et''.

  \item あの \ruby{漢字}{かんじ}は \ruby{何}{なん}と \ruby{読}{よ}むんですか。\\
    ・・・「きんえん」です。\\
    たばこを \underline{\ruby{吸}{す}うな}と いう \ruby{意味}{いみ}です。\\
    Que signifie ce kanji?\\
    Il signifie ``interdit de fumer''.
  \end{itemize}

\end{CJK*}
\end{document}
