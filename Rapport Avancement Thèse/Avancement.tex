\documentclass[10pt]{article}

\usepackage[utf8]{inputenc} %Prise en charge des accents
\usepackage{csquotes}
\usepackage[english, french]{babel} %Charge les langues
\usepackage{color} %Charge les couleurs
\usepackage{fancyhdr} %Utilisation des entêtes
\usepackage{lastpage} %Script pour avoir le nombre de pages
\usepackage{graphicx} %Permet d'inclure des images
%\usepackage{makeidx} 
\usepackage{amsmath}
\usepackage{xspace} %Espaces intelligents
\usepackage{caption}
\usepackage{subfig}
\usepackage{multirow}
\usepackage{tabularx}
%\usepackage{picins} %Permet de faire des sous-figures
\usepackage{tikz} %Permet de dessiner
\usepackage[backend=bibtex]{biblatex}
\addbibresource{papers.bib}

\usepackage[top=3.5cm, bottom=3.5cm]{geometry}

\author{Léo \textsc{Baudouin}}
\title{Avancement Thèse}

\begin{document}

%\maketitle
\begin{center}
\begin{Huge}
Avancement Thèse
\end{Huge}

\vspace{3mm}
\begin{Large}
Léo \textsc{Baudouin}
\end{Large}

\vspace{3mm}
\today
\end{center}


\section{Vision}
\subsection{Vision classique}

A partir du livre de \citeauthor{Hartley03Book} \citetitle{Hartley03Book}, j'ai écrit les fonctions en C++ (avec OpenCV) nécessaire pour la reconstruction 3D.
C'est à dire : 
\begin{itemize}
\item RANSAC
\item Triangulation
\item Calcul de la matrice Fondamentale/Essentielle
\item Extraction des matrices $\mathbf{R}$ et $\mathbf{t}$
\item Ajustement de faisceaux
\end{itemize}
J'ai essayé plusieurs solutions lorsque c'était possible (SVD, pseudo-inverse, Levenberg-Marquardt, \dots), ou plusieurs méthodes (triangulation).
J'ai essayé également le tenseur trifocal sans succès.

J'obtiens des résultats satisfaisant à partir d'images simulées :
\begin{figure}[ht]
\begin{center}
\subfloat[Nuage simulé]{\label{fig:create}\includegraphics[width=0.45\linewidth]{images/createimage.png}}
\subfloat[Nuage recréé]{\label{fig:build}\includegraphics[width=0.45\linewidth]{images/buildfromvideo.png}}
\caption{L'image~\ref{fig:create} représente le nuage de points simulé ainsi que les différentes prises de vues, l'axe de la caméra ($\vec{z}$) est en bleu. Dans l'image~\ref{fig:build} on peut voir le résultat obtenu lors de la création du nuage à partir des images précédemment simulées.}
\end{center}
\end{figure}

\subsection{Vision Omnidirectionnelle}

J'ai écrit les mêmes fonctions dans le cas omnidirectionnelle.
Cependant j'ai des problèmes à estimer correctement $\mathbf{R}$ et $\mathbf{t}$ entre deux vues successives.
La méthode présentée dans \citetitle{Puig11PhD} de \citeauthor{Puig11PhD} a été implémentée.
Celle-ci donne de bons résultats quand la caméra se déplace en ligne droite mais dès qu'elle tourne, l'algorithme ne converge plus vers la solution.
Ceci vient des erreurs de calculs sur les angles de rotation, mais je ne parviens pas à corriger ce problème.

\subsection{SoViN}

J'ai testé la librairie SoViN, mais je n'ai pas encore pu l'utiliser au sein de mon programme. Pour l'instant SoViN est fait pour fonctionner de manière autonome, en utilisant une interface conçue par J.~\textsc{Courbon}.
Je viens juste d'avoir la version~2.0, et je vais essayer de participer au développement afin d'implémenter de nouvelles fonctions utiles dans le cadre de ma thèse.

\section{Fusion de cartes}

Afin de mettre en place la fusion de carte, je voulais avoir un a priori sur la position relative des deux nuages de points.
J'ai donc essayé avec l'algorithme ICP, mais celui-ci nécessite un trop grand nombre de point en commun (plus de 70\% selon mes tests).
D'autre part il est trop sensible à l'échelle des nuages de points.
Si l'échelle n'est pas exactement la même, l'algorithme ne permet pas d'obtenir la solution.

En connaissant des points communs aux deux cartes, je peux actuellement fusionner les nuages de points 3D, et ainsi créer une carte globale qui servira pour tous les robots.
Mais le problème consiste à trouver le moyen d'avoir une liste de points en correspondance.

J'ai voulu testé la librairie ScaViSLAM (voir sur OpenSLAM.org) qui proposent plusieurs algorithmes pour le SLAM.
Mais très peu sont adaptés à la vision.
J'ai également contacté H.~Korrapati afin d'avoir ses codes-sources utilisés pour la fermeture de boucles en utilisant uniquement des descripteurs SIFT.
Je vais tenter de l'utiliser pour mettre les points en correspondances, même si cet algorithme risque de prendre beaucoup de temps avec un très grand nombre de points (plusieurs robots avec de grandes cartes)

\section{Navigation}

J'ai lu de nombreux articles concernant la navigation de robot mobile utilisant des cartes pré-construites.
Je n'ai pas encore commencé à programmer cette partie, je vais commencer dans peu de temps.
Je suis en train de m'intéresser aux librairies ROS et MRPT pour voir si elles pourraient être utilisées.

\section{Conclusion}

Je sais que j'ai pris énormément de retard en reprogrammant les fonctions de base pour la vision.
Cependant j'avais besoin de comprendre réellement comment elles fonctionnaient.
L'échec de la reconstruction dans le cas omnidirectionnel est un problème majeur pour moi, car je n'ai pas réussi à trouver des librairies existantes permettant de résoudre ce problème.
Sachant que le laboratoire utilise couramment ce genre de fonctions, j'espère pouvoir trouver rapidement ce que je recherche.

En attendant je vais continuer de travailler avec des nuages de points simulés pour mettre en place des algorithmes de navigation et de mise en formation de flotte de robots.
Je dois également tester différentes librairies (ROS, MRPT, \dots) afin de ne pas avoir à ré-écrire un grand nombre de fonctions.

\end{document}
