\section{R\'esultats}
\label{sec:result}

\subsection{Commande}
Le programme utilisé sur le micro-controleur permet de choisir le gain appliqué sur la consigne.
 Sur l'exemple de la figure~\ref{fig:gain}, avec un gain unitaire la réponse a un retard assez important ce qui ne permet pas toujours de terminer le mouvement.
 Avec un gain de 3, nous pouvons voir que la réponse oscille car le système tente de rattraper son retard trop rapidement.
 Dans notre cas à 8 bars, un gain de 2 est alors retenu.
 
\smallfig{1}{images/gain}{Influence du gain lors d'un déplacement de 580 incréments en 10 secondes.}{fig:gain}
  
En fonction de la pression utilisée dans le système final il faudra adapter ce gain.
Sachant que la pression ne sera pas constante, il faudra créer une table de conversion ou déterminer une fonction de conversion.
N'ayant pas obtenu de baromètre précis je n'ai pas été en mesure de réaliser ceci.
\saut 
 
 
 Dans la figure~\ref{fig:stop}, le bras a été bloqué à la deuxième seconde du mouvement.
 Le retard est donc de plus en plus important impliquant une commande également plus importante.
 Le bras étant bloqué, la pression augmente au sein du piston.
 
 Sur le deuxième graphique, on peut voir une pression négative et qui diminue lors du blocage.
 Ceci est du à la partie de code permettant la lecture du capteur de pression qui utilise une table de conversion qui ne correspond pas au capteur qui m'a été fourni.
 
  \smallfig{1}{images/stop}{Blocage du bras lors d'un mouvement.}{fig:stop}
  
 D'autre part nous pouvons remarquer que la pression oscille sur le segment de temps $[2.5:5.5]$ alors qu'elle ne devrait pas, car tous les corps solides sont bloqués.
 Ce ph\'enom\`ene est due \`a la pompe hydraulique utilis\'ee qui ne garantie pas une pression constante.
 
\subsection{Programmes}
 
\subsubsection*{"Temps r\'eel"}

 Les micro-controleurs utilis\'es afin de commander les articulations du robot sont suceptibles de subir des perturbations dans le suivi du temps.
 Si les calculs \`a r\'ealiser sont trop longs, la valeur du temps fourni par le micro-controleur est allong\'e.
 Ce d\'efaut pourrait avoir des cons\'equences grave dans certain cas n\'ecessitant un contr\^ole du temps pr\'ecis.
 
 
 \subsubsection*{Pr\'ecision de lecture}
 
 Le capteur de position est un capteur absolu \`a 4096 positions.
 La valeur envoy\'ee est donc tr\`es fiable, avec une pr\'ecision de 0.175~degr\'es.
 Cependant afin de lire la valeur, le micro-controleur doit calculer le ratio $\frac{\text{Temps au niveau}}{\text{P\'eriode}}$.
 Le calcul de ces temps fait apparaitre certaines valeurs aberrantes qui doivent \^etre filtr\'ees.
 
 La valeur de la pression est quand \`a elle moins pr\'ecise.
 Le capteur\footnote{PX26-250DV} fourni une valeur qui d\'epend de sa tension d'alimentation.
 A l'aide du g\'en\'erateur utilis\'e nous ne pouvons garantir une pr\'ecision inf\'erieure \`a 1 volt ce qui implique une erreur de 5 \`a 10\% sur la valeur d\'elivr\'ee.
 
La somme des erreurs dans chacunes des articulations pourra fausser le r\'esultat final et rendre certains objectifs irr\'ealisables, comme celui d'utiliser \emph{Junior} afin d'attraper et de casser des {\oe}ufs.
 
 \subsection*{Temps de transfert}
 
Le robot est supervis\'e par un programme en Visual Basic sur un ordinateur portable.
Il faut donc communiquer entre cette entit\'e et le micro-controleur.
Ceci est r\'ealis\'e via un port s\'erie simul\'e \`a 19200~bauds sur le micro-controleur et connect\'e sur un port USB de l'ordinateur.
Toutes les commandes sont donc envoy\'ees sous forme d'une chaine de caract\`eres comportant au minimum une action (une lettre) et trois param\`etres (flotants).
Cette technique oblige le micro-controleur \`a renvoyer une confirmation de r\'eception des messages, et ralenti donc \'enorm\'ement la vitesse de r\'ealisation des t\^aches.

