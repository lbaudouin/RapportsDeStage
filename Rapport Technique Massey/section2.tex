\section{G\'en\'eration de la commande}
\label{sec:command}

Le robot devant au final \^etre utilis\'e dans un environnement destin\'e \`a l'\^etre humain, il faut que les gestes r\'ealis\'es soient suffisament souples pour ne pas d\'eranger ou faire peur aux personnes aux alentours.

On choisira donc un mouvement sans acc\'el\'eration brutale ce qui implique une vitesse continue.
Le type de courbe retenu pour la vitesse est du type parabolique.

\subsection{Param\`etres}

Les param\`etres importants pour la commande sont principalement la distance \`a parcourir ainsi que le temps de parcours d\'esir\'e. 
Pour cel\`a on a besoin des variables suivantes :\\

\begin{tabular}{l l}
  \begin{minipage}{0.5\linewidth}
    \begin{itemize}
    \item Distance : $d$
    \item Temps : $t_f$
    \item Temps court : $t_s = 0.75 \times t_f$
    \end{itemize}
  \end{minipage}
  &
  \begin{minipage}{0.5\linewidth}
    \begin{itemize}
    \item Position de d\'epart : $p_{d\acute{e}part}$
    \item Position courante : $p_{courante}$
    \item Gain : $K$
    \end{itemize}
  \end{minipage}
\end{tabular}
\vspace{5mm}

Le param\`etre $t_s$ permet d'anticiper le retard dans la commande. Effectivement, les temps de communications entre l'ordinateur embarqu\'e et le micro-controleur sont difficilement estimables. D'autre part il existe \'egalement un laps de temps non n\'egligeable entre la r\'eception de la commande par le micro-controleur et l'application r\'eelle de celle-ci sur le servo-moteur.
Si ces temps \'etaient finalement estim\'es nous pourrions avoir $t_s = t_f - t_{retard}$.

\vspace{10mm}
Fonction de vitesse souhait\'ee avec une vitesse nulle au point de d\'epart :
\begin{equation}
  v(t) := \frac{a.t^{2}}{2} + b.t
\end{equation}

En int\'egrant la fonction de vitesse on a la fonction de position impos\'ee :
\begin{equation}
  p(t) := \frac{a.t^{3}}{6} + \frac{b.t^2}{2} + p_{d\acute{e}part}
\end{equation}

En imposant que les conditions suivantes soient respect\'ees :

$$
  v(t) = \left\{
  \begin{array}{l l}
    0 & \quad \text{si $t=0$}\\
    0 & \quad \text{si $t=t_s$}\\
    \frac{a.t^{2}}{2} + b.t & \quad \text{sinon}\\
  \end{array} \right.
$$
$$
  p(t) = \left\{
  \begin{array}{l l}
    0 & \quad \text{si $t=0$}\\
    d & \quad \text{si $t=t_s$}\\
    \frac{a.t^{3}}{6} + \frac{b.t^2}{2} + p_{d\acute{e}part} & \quad \text{sinon}\\
  \end{array} \right.
$$

On obtient :
$$
  a = -12 \times \frac{d}{{t_s}^3}~,~b=6\times \frac{d}{{t_s}^2}
$$

La commande g\'en\'er\'ee en fonction de l'erreur est alors :
\begin{equation}
  c(t) := K \times ( position(t) - p_{courante})
\end{equation}

Exemple :\\
\smallfig{0.85}{images/commande}{Exemple de consigne pour $d=2$, $t_f=4$ et $K=2.5$}{fig:command}
\newpage


\subsection{Mise \`a l'\'echelle}

\`A partir d'une valeur dans $\Re$ nous devons commander la valve controlant l'articulation.
Le servo-moteur utilis\'e pour commander la vanne fonction avec des valeurs de commande comprisent entre 0 et 250, avec une plage [100:150] ou la consigne est trop faible pour permettre au servo-moteur de se d\'eplacer.
\vspace{5mm}

On doit donc corriger la commande %\emph{output}
pr\'ec\'edemment calcul\'ee :

\begin{enumerate}
%\item {\bf output *= -1;} \textcolor{gray}{//Changement de direction}
\item Valeur minimale sens positif\\
{\bf commande$=$commande$>$0?commande$+$25:commande;}
\item Valeur minimale sens n\'egatif\\
{\bf commande$=$commande$<$0?commande$-$25:commande;}
\item D\'eplace la ligne m\'ediane \`a 125\\
{\bf commande$+=$125;}
\item Borner la valeur dans [0:250]\\
{\bf commande$=$commande$<$0?0:commande$>$250?250:commande;}
\end{enumerate}

\smallfig{0.75}{images/crop-fr}{Consigne envoy\'ee au servo-moteur (l'\'echelle n'est pas respect\'ee)}{fig:scale}

\subsection{\emph{Chained Move}}
\label{sub:chained}

Une fois que la commande pr\'ecise fut \'elabor\'ee, \rory m'a demand\'e de construire un nouveau type de commande : "\emph{Chained Move}".
Elle consiste en une commande aveugle avec pour param\`etres la dur\'ee de parcours et la distance (sign\'ee) \`a parcourir.
L'objectif est d'atteindre la position finale au moment pr\'evu mais cette fois-ci avec une vitesse finale quelconque. 
Cette commande sera lanc\'e tr\`es rapidement, toutes les $100~ms$ environ.
Cette fonction permet de changer de consigne tr\`es r\'eguli\`erement afin de s'adapter aux mouvements du robot (retards, collisions,...).
Afin que la position finale soit r\'eellement atteinte, une commande de pr\'ecision (voir~\ref{sec:command}) sera r\'ealis\'ee en fin de mouvement.

Si le syst\`eme est vraiment aveugle, nous ne pouvons pas avoir la vitesse en d\'ebut de trajectoire.
Dans la figure~\ref{fig:chained}, la forme du signal choisi est une droite passant par l'origine pour chacunes des dix consignes cons\'ecutives.
\smallfig{0.65}{images/chained}{Nouvelle commande propos\'ee pour controler le bras.}{fig:chained}

En connaissant la vitesse au d\'ebut de trajectoire, la forme du signal pourrait \^etre au mieux continue mais non lisse si on conserve une commande lin\'eaire.
L'utilisation de courbes de b\'ezier ou de splines rendrait le mouvement plus agr\'eable pour un environnement peupl\'e d'humains.

\newpage
