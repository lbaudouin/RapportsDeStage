\newpage
\section{Conclusion}
\label{sec:conlusion}

Durant ce stage, la majeur parti de temps a \'et\'e consacr\'e \`a la compr\'ehension du code source existant.
Effectivement, le code qui m'a \'et\'e fourni pour le projet en cours n'\'etait pas suffisament clair pour \^etre facilement utilisable.
On y retrouve les d\'efauts suivants :
\begin{enumerate}
\item nom des variable ambig\"ue: \emph{temp}, \emph{alpha}, \emph{beta}, ...
\item d\'eclaration d'une partie des fonctions et des variables globales dans le fichier source et non dans le header
\item utilisation de pointeurs sur des variables globales uniques dans les arguments des fonctions internes
\end{enumerate}
Tout ceci refl\`ete un manque de rigueur et un manque de m\'ethodes de programmation qui engendrent une perte de temps in\'evitable pour tous les nouveaux arrivants sur le projet ou pour la maintenance.

\subsection{Travaux réalisés}
\label{sub:travauxrealises}
Les objectifs atteints lors de ce stage sont :
\begin{itemize}
\item G\'en\'eration de la g\'eom\'etrie inverse en Visual Basic
\item Compr\'ehension du programme fourni
\item Controle en position d'une articulation
\item Lecture de la pression dans le muscle en "temps r\'eel"
\end{itemize}

\subsection{Travaux restants}
\label{sub:travauxrestant}
Les points restants \`a voir sur ce projet :
\begin{itemize}
\item Controle en force de l'articulation
\item Nettoyage/Optimisation du code source en C sur le micro-controleur
\item Proposer une migration du VB au C++ pour la partie supervision
\end{itemize}

\subsection{Conclusion générale}
\label{sub:conclusion}

Le fait d'arriver sur un sujet sans qu'on me l'ai pr\'esent\'e auparavant m'a l\'eg\`erement perturb\'e au d\'ebut du stage.
Je n'ai donc pas pu choisir concr\`etement et en connaissance de cause ce sur que j'allais travailler pendant les 5 mois suivants.
Je n'aurais sans doute pas fait le choix de ce sujet s'il m'avait \'et\'e pr\'esent\'e avant.

D'autre part, une fois sur place, il a \'et\'e pour moi tr\`es difficile de me faire comprendre et donc d'obtenir les logiciels ou le mat\'eriel dont j'avais besoin pour les exp\'eriences, ou l'aide n\'ecessaire pour avancer.
Ce d\'esagr\'ement vient principalement de mon faible niveau en anglais oral qui n'a pas rendu mon int\'egration facile au sein de l'\'equipe.

L'impression que m'a laiss\'e l'\umassey est un sentiment de manque de rigueur g\'en\'eral particuli\`erement flagrant au niveau informatique. Je pense que les projets devrait se focaliser sur un point particulier plutôt que sur un robot complet.
%Les projets r\'ealis\'es ne sont en g\'en\'eral que du bricolage, 
%Compar\'e \`a un projet dans un laboratoire fran\c{c}ais, il y a un manque de rigueur permanent surtout au niveau informatique.

\saut
Concernant le projet \emph{Junior}, je pense que ce robot n'ai pas viable \`a long terme.
Principalement parce qu'il existe d'autres robots tels que ASIMO ou HRP-2 qui sont capables de faire au moins aussi bien que les objectifs vis\'es par M. \rory.
Seule la charge effective suppos\'ee de \emph{Junior} est sup\'erieur au maximum admissible par ces robots humano\"{\i}des.
\emph{Junoir} devrait \^etre capable de soulever $30~kg$ alors que HRP-2 est limit\'e \`a $20~kg$.
D'autre part, l'utilisation d'une pompe hydraulique tr\`es bruyante embarqu\'ee dans la base roulante est tr\`es g\'enante pour un robot destin\'e \`a vivre dans un domicile.
%Elle impose \'egalement des batteries tr\`es volumineuses
Lors de mon d\'epart, le robot n'\'etait pas encore exploitable car il \'etait impossible de maintenir la pression \`a cause des nombreuses fuites dans le circuit hydraulique. 
