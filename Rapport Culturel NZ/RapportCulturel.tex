\documentclass[11pt,journal]{RapportFR}
%\documentclass[11pt,twocolumn]{article}

%\usepackage{frcursive}

\newcommand{\Nz}{Nouvelle-Z\'elande\xspace}
\newcommand{\nzs}{n\'eo-z\'elandais\xspace}
\newcommand{\nze}{n\'eo-z\'elandaise\xspace}
\newcommand{\nzes}{n\'eo-z\'elandaises\xspace}
\newcommand{\Nzs}{N\'eo-Z\'elandais\xspace}
\newcommand{\PM}{Palmerston North\xspace}
\newcommand{\ids}{\^{\i}le du Sud\xspace}
\newcommand{\idn}{\^{\i}le du Nord\xspace}

% correct bad hyphenation here
%\hyphenation{op-tical net-works semi-conduc-tor}
\usepackage[center]{caption}

\begin{document}

\title{Rapport Culturel - \Nz 2011}
\author{
  \begin{huge}L\'{e}o \bsc{Baudouin}\end{huge}\\
  \textit{MMS - M\'ecatronique}\\
  \today
}

\maketitle

\begin{abstract}
Ce rapport est un compte-rendu des observations sur la \Nz et les \Nzs. Il r\'esumera une partie de leur histoire, leur mode de vie et les diff\'erents grands points que j'ai pu remarquer lors de mon s\'ejour dans la ville de \PM ainsi que pendant les diff\'erentes visites dans le pays.
\end{abstract}

\begin{keywords}
Stage \`a l'\'etranger, \Nz, Massey University, rapport culturel.
\end{keywords}


%\begin{cursive}
\section{Introduction}
\label{sec:intro}

\RapportFRstart{P}{ourquoi} choisir la \Nz et pas un autre pays ? 
C'est une question que je me suis pos\'ee plusieurs fois lorsqu'il a fallu choisir mes deux stages pour l'ann\'ee \`a l'\'etranger. 
Je m'étais imposé une contrainte : absolument partir dans un pays anglophone afin d'am\'eliorer le plus rapidement possible mon niveau en anglais.

Mon id\'ee principale \'etait de partir le plus loin possible afin de changer d'air pendant un an. La \Nz \'etait le pays qui r\'epondait le mieux \`a ce crit\`ere tout en proposant des stages dans le domaine de la m\'ecatronique. 
L'autre motivation r\'esidait dans la beaut\'e du pays à découvrir.
Je me suis donc basé sur la réputation du pays, qui dit que l'on visite la \Nz essentiellement grâce aux milliers de randonnées qu'elle permet au travers de magnifiques paysages.
%R\'eput\'e pour proposer de nombreuses randonn\'ees, ce pays \`a l'art de d\'evoiler ses sompteux paysages apr\`es quelques heures de marche.

Pour ce voyage \`a l'autre bout du monde, ma compagne a d\'ecid\'e de me suivre et de partager mon exp\'erience, j'utiliserai donc le pluriel quand les reflexions refl\`eteront ce que nous avons v\'ecu ou pens\'e tous les deux.

Apr\`es mon stage au Japon o\`u nous n'avions pas beaucoup eu l'occasion de parler avec des Japonais, nous avons d\'ecid\'e de nous int\'egrer d'avantage en choisissant de vivre en colocation avec deux couples \nzs. Ceci nous a permi de nous habituer plus rapidement \`a l'accent et au mode de vie.

Sur les 4 mois et demi de stage nous avons pr\'evu de rester les 4 premiers mois sur l'\idn en en visitant une région tous les week-ends. Les deux derni\`eres semaines seront d\'edi\'ees \`a un voyage dans l'\ids, qui regroupe \`a elle seule plus de 75\% des plus beaux paysages de la \Nz.

% --------------------------------------------------------------------

\section{Histoire de la \Nz}
\label{sec:NZ}

L'histoire de ce pays est l'une des histoires les plus courtes du monde, car il s'agit d'un des derniers territoires découverts par l'Homme. En effet, les M\=aori y sont arrivés entre 1050 et 1300, tandis que les Européens n'y sont arriv\'e qu'\`a partir de 1642.
\fig{images/New_Zealand}{Drapeau de la \Nz. \textit{New~Zealand} (Anglais), \textit{Aotearoa} (M\=aori).}{fig:drapeau}

\vspace{-10mm}
\subsection{Les M\=aori}
\label{sub:maori}

Les M\=aori sont le regroupement de populations polynésiennes autochtones de Nouvelle-Zélande. Ils s'y seraient installés par vagues successives à partir du VIII\up{e} siècle.
%Ils sont, à l'aube de l'an 2000, plus de 600 000 auxquels il faut ajouter une diaspora d'environ 90 000 personnes dont une grande majorité vit en Australie.
Premiers habitants de la Nouvelle-Zélande, ils représentent aujourd'hui 15\% de la population, soit environ 600 000 personnes.

Les M\=aori sont ``\textit{la dernière communauté humaine majeure de la terre qui n’ait pas été touchée ni affectée par le vaste monde}''\footnote{Michael King, \emph{History of New Zealand}}.

Ils parlent une langue composée avec 5~voyelles et 10~consonnes, le m\=aori. Elle est, avec l'anglais, une langue officielle du pays.
Cette langue est \'egalement utilis\'ees pour la plupart des noms des villes/villages/rues du pays, \`a l'exception des grandes villes.

Cependant, les M\=aori sont de plus en plus marginalis\'es et se retrouve de nos jours progressivement concentr\'es dans la banlieue sud d'Auckland.
Comme beaucoup de minorit\'es, ils ont une image de pauvres, de brutes ou de voleurs aux yeux de beaucoup de \Nzs, bien que la r\'ealit\'e soit évidemment toute autre.

\subsection{Les Colons}
\label{sub:colons}

L’installation des Européens en \Nz est relativement récente. La rencontre avec les M\=aori n'a pas été très facile.
C'est en 1642 que la Compagnie hollandaise des Indes orientales envoie Abel Tasman qui aborde l'\ids de la \Nz.
Il repart aussitôt face à l'hostilité des autochtones.

Les premiers explorateurs Européens (Abel Tasman et le capitaine James Cook\footnote{qui a visité la \Nz pour la première fois en 1769}) ont ensuite rapporté leur rencontre. Ils décrivent les M\=aori comme une race de guerriers féroces et fiers. 
Des conflits inter-tribaux se produisaient fréquemment à cette période, et les vainqueurs rendaient esclaves les vaincus voire parfois les dévoraient.

%Dès le début de l’année 1780, les M\=aori ont eu des contacts avec les chasseurs de baleines et de phoques. Certains se sont même fait embaucher sur des navires étrangers. 
%Un flot continu de prisonniers Australiens en fuite et de déserteurs provenant des navires de passage a également exposé la population des autochtones \nzs aux influences extérieures.

A partir de 1769, les anglais se sont peu à peu implanté sur le territoire.
Plusieurs guerres ont ensuite éclaté entre les M\=aori et les colons.
En 1840 le \emph{Traité de Waitaingi} est signé créant ainsi la \Nz et garantissant les droits des M\=aori.
Elle restera cependant sous l'authorité de la Grande-Bretagne tout en continuant de faire co-habiter M\=aori et colons.
C'est seulement en 1947 qu'elle devient entièrement indépendante.

De nos jours, la population issue de la descendance des colons représente 86\% de la population totale de \Nz.
Les \emph{kiwis}\footnote{terme non péjoratif et largement employé pour désigner les \Nzs} sont donc de plus en plus éloignés de la culture apport\'ee par les premiers habitants des \^{\i}les.
\\
\\

\section{Palmerston North}
\label{sec:pm}

\subsection{La ville}
\label{sub:city}

Avec une population de 81.300~habitants (\emph{les 2/3 de celle de Clermont-Ferrand}) et une superficie importante de 356~km\up{2} (\emph{7 fois celle Clermont-Ferrand}), \PM a les proportions d'une grande ville tr\`es faiblement peupl\'ee. Elle est la 8\up{\'eme} plus grande ville du pays.

La faible densit\'e (\emph{249.4~hab/km\up{2}}) s'explique par l'absence d'immeuble ou d'HLM.
La majorit\'e des habitations sont des maisons de plain-pied poss\'edant la plupart du temps un terrain. Tr\`es peu d'entre elles ont un \'etage, car elles sont construite afin de pouvoir résister à un tremblement de terre important.
Ce type de ville a le m\'erite de proposer des maisons tr\`es espac\'ees qui rendent le cadre de vie très calme.

Cependant, la ville \'etant relativement r\'ecente, la construction ressemble \`a une ville typiquement am\'ericaine avec de tr\`es longues rues parfaitement perpendiculaires comme on peut le voir sur le plan figure~\ref{fig:map}.

 
\fig{images/Map}{Plan de \PM}{fig:map}

Seul l'université de Massey (voir~\ref{sub:massey}) est a l'écart au sud et n'a donc pas été construite sur le m\^eme modèle.
Ce type de ville ne pr\'esente que tr\`es peu d'int\'er\^ets architecturaux.
Comme la plupart des villes de l'\idn, \PM n'est pas une ville tr\`es agr\'eable \`a visiter due à la monotonie des décors.
Seul un \emph{square} (figure~\ref{fig:palmy}) et quelques parcs viennent embellir la ville avec un peu de verdure.

\fig{images/PalmerstonNorth}{Vu sur le \emph{square} de Palmerston North}{fig:palmy}

%La position de \PM dans l'\^\i{}le du nord permet aux \emph{Palmerstonians} d'acc\'eder rapidement \`a des zones int\'eressantes comme Wellington, Napier, National Park.
%\smallfig{0.4}{images/Localisation}{Emplacement de Palmerston North sur l'\^\i{}le du Nord, \`a 160km au dessus de Wellington}{fig:loca}

Malgr\'e son apparence plut\^ot calme, \PM n'est pas d\'epourvu de violences et de vols notamment \`a cause de quelques gangs agissant au niveau du centre ville.
Les \Nzs nous ont fortement déconseillé de circuler \`a pied le soir dans les rues de \PM à cause du \emph{racket} et des possibles agressions.

Le vol de voiture est également une pratique assez courante gr\^ace \`a la souplesse du march\'e des v\'ehicules d'occasion, qui facilite la revente du voitures volées. Un panneau revient donc tr\`es souvent : ``\emph{lock it or lose it}''.

\vfill

\subsection{Massey University}
\label{sub:massey}

La plus importante universit\'e de \PM est l'universit\'e de Massey, elle se situe au sud de la ville \`a environ 10~minutes en bus.
L'universit\'e est d\'ecoup\'ee en trois antennes dont une \`a Auckland et une \`a Wellington. Elle regroupe un peu plus de 36\,000~\'etudiants dont 9\,000 sur le campus de \PM..

\smallfig{0.35}{images/Massey_University}{Logo de l'universit\'e de Massey}{fig:massey} \vspace{-5mm}

Bien que l'universit\'e soit publique, le prix des \'etudes est assez important. Il faudra d\'ebourser pr\`es de 150\euro\xspace par mati\`ere pour obtenir 15~cr\'edits. 
Sachant qu'il faut 60~cr\'edits par semestre, un \'etudiant devra au final avancer pr\`es de 1200\euro\xspace par an pour ses \'etudes.
Beaucoup de jeunes n'ont pas les moyens financiers pour pouvoir s'inscrire \`a Massey University et sont donc contraints d'int\'egrer directement le monde actif.

%Les aides du gouvernement sont assez importantes, mais malgré ceci, on peut constater un certain retard technologique dans plusieurs domaines notamment dans celui de l'usinage o\`u l'on ne retrouve que des machines manuelles, ou de la robotique.
%D'autre part les robots d\'edi\'es pour la recherche ne sont pas achet\'es mais cr\'e\'es au sein de l'universit\'e.
%Ceci a pour cons\'equence un temps d'attente important avant de pouvoir effectuer des recherches concr\^etes.


Malgr\'e la r\'eputation de l'universit\'e, on remarque un manque de rigueur scientifique permanent, que ce soit en informatique, en m\'ecanique ou en biologie ...
On trouvera donc des mod\`eles de robots n\'egligeant des d\'ecalages de plusieurs centim\'etres entre deux axes, des programmes informatiques involontairement obfusqu\'es\footnote{technique rendant un code-source difficilement comprehensible} ou des calibrations approximatives pour des outils de pr\'ecision.
Les nombreux stagiaires fran\c{c}ais d\'eplorent ce manque de rigueur au quotidient dans les domaines scientifiques.

Pour la robotique, les robots sont cr\'e\'es au sein m\^eme de l'universit\'e.% sur les machines outils manuelles de l'atelier. 
Il faut donc, pour chaque nouveau robot, concevoir tout le syst\`eme depuis la m\'ecanique jusqu'aux programmes de supervision.
Ceci oblige les chercheurs \`a se consacrer \`a la totalit\'e du robot et non juste \`a une partie qui pourrait \^etre un sujet de recherche \`a part enti\'ere (comme la vision). 

D'autre part, la formation des \'el\`eves passe obligatoirement par un stage de recherche qui se d\'eroule en parall\`ele des cours \`a l'universit\'e.
Certains professeurs sont contre cette pratique, jugeant que le stage de recherche est souvent bacl\'e par manque de temps. Ils souhaitent que ce stage disparaissent pour les \'el\`eves qui ne sont pas attir\'es par le domaine de la recherche scientifique. Pour les autres, ce stage devrait \^etre remplacer par un stage \`a temps plein sur le domaine de recherche concern\'e.

%\begin{enumerate}
%\item R\'eparti dans toute la \Nz
%\item Prix
%\item Retard technologique
%\item Abscence de m\'ethodologie
%\item Rigueur scientifique : Informatique, biologie
%\end{enumerate}

% --------------------------------------------------------------------

\section{Mode de Vie}
\label{sec:life}

%magasin ferment tot

\subsection{Alimentation}
\label{sub:food}

%Il ne faut pas oublier que la \Nz est culturellement très proche de la Grande Bretagne. Donc niveau nourriture, le traditionnel pâtes/jambon français n’est pas répandu du tout, c’est plutôt bacon et petit déjeuner anglais.
La cuisine \nze a longtemps été basée sur les \emph{fish and chips}, le \emph{meat pie} et le \emph{pudding} directement débarqués d’Angleterre.
Cependant les \Nzs sont de plus en plus influenc\'es par le mode de vie des am\'ericains. On trouve donc maintenant \'enormement de \textit{fast-food} dont les jeunes sont de grands adeptes.
On estime que la population \'etudiante consomme dans ce genre de ``restaurant'' en moyenne 50\% du temps.
Ceci explique un taux important de personnes en surpoids : 68.4\% de la population, et un taux d'ob\'esit\'e de 26.5\%. Les personnes ob\`eses repr\'esentent donc un cinqui\`eme de la population \nze, contre seulement un dizi\`eme en France.
La \Nz vient se classer, en 2007, au 2\up{nd} rang mondial concernant les personnes en surpoids, juste derri\`ere les \'Etats~Unis.

De nombreux fruits sont cultiv\'es dans la r\'egion de Napier et de la \emph{Bay of Plenty}, comme des pommes, des raisins, des kiwis, des pêches ou des avocats qui permettent de toujours avoir des fruits frais et m\^urs. Cependant la \Nz exporte une tr\`es grande partie de ses r\'ecoltes, ils profiteront donc plus aux pays comme le Japon ou l'Australie.


\subsection{Transports}
\label{sub:transport}

Les syst\`emes de transport en commun intercit\'e ne sont pas encore tr\`es d\'evelopp\'es en \Nz.
Il y a tr\`es peu de trains et le prix des billets est trop cher pour \^etre envisageable comme unique moyen de transport.
Un r\'eseau de bus \emph{low-cost} permet de relier les principales grandes villes du pays. Il faudra compter 25\$ (soit 15\euro) pour faire les 650km s\'erarant Wellington et Auckland. 

Les transports intracit\'es se limitent \`a un r\'eseau de bus tr\`es peu pr\'esent quand on ne se trouve pas dans l'une des deux capitales Auckland\footnote{capitale \'economique} et Wellington\footnote{capitale officielle}.
Ceci a pour cons\'equence que la plupart des \Nzs vont travailler en voiture, bien que les villes essayent de promouvoir l'utilisation des v\'elos, celui-ci reste tr\`es limit\'e \`a cause de la taille importante des villes.

Malgr\'e des taxes importantes pour le financer, le r\'eseau routier des \^{\i}les n'est pas en tr\`es bon \'etat.
D\`es que l'on s'\'eloigne des principaux axes, certaines routes ne sont plus beaucoup entretenues et ne sont par endroit m\^eme pas goudronn\'ees sur plusieurs kilom\`etres.
D'autre part on regrettera l'abscence d'autoroute \`a grande vitesse.
Avec une vitesse maximale de 100 km/h dans tout l'archip\`ele, il faut compter plus de 10 heures de route pour traverser enti\`erement l'\idn.

\subsection{Sports}
\label{sub:sport}

%Parmi les autres sports très populaires, on trouve le~cricket, le~football, le~rugby à XIII, le~basket-ball, le~netball et le~Boulingrin, ainsi que le~golf, le~tennis, le~cyclisme, le~hockey sur gazon, le~ski, le~snowboard, le~softball et plusieurs sports nautiques, dont le~surf, le~nautisme, le~kayak, le~surf lifesaving et l'aviron. Elle est également reconnue pour son bon ratio médailles-population aux Jeux olympiques et du Commonwealth.

Les sports les plus populaire en \Nz sont le rugby et le cricket, mais on trouve des sports moins connus comme le netball, le softball ou le boulingrin.

\vspace{2mm}
\subsubsection{Rugby}
\label{subsub:rugby}

Le rugby à XV est très étroitement lié à l'identité nationale néo-zélandaise. L'équipe de Nouvelle-Zélande de rugby à XV, surnommée les ``\textit{All~Blacks}'', a les meilleures statistiques de victoires de toutes les équipes nationales.
% Elle a accueilli la première Coupe du Monde de Rugby à XV, qu'elle remporta.

L'un des \'ev\`enements les plus importants du pays en 2011 est sans doute la Coupe du Monde de Rugby, suivi par la totalit\'e des \Nzs. Pendant plus d'un mois le rugby \'etait le seul sujet de conversation possible.
Le soir de la finale il ne faisait pas bon d'\^etre fran\c{c}ais dans certains bars \`a cause des tensions entre les suporters rivaux.
Ceux qui ont regard\'e les matchs ont pu voir le \emph{Haka} (voir figure~\ref{fig:haka}) qui est une danse guerrière traditionnelle m\=aori, elle est exécutée par les joueurs juste avant le début de chaque match afin d'intimider les adversaires.
\fig{images/haka}{Haka r\'ealis\'e par l'\'equipe des \emph{All Blacks}.}{fig:haka}

\vspace{-5mm}
\subsubsection{Cricket}
\label{subsub:cricket}

Le cricket est considéré comme le principal sport estival de la Nouvelle-Zélande et l'équipe de Nouvelle-Zélande de cricket (surnommée les ``\textit{Black~Caps}''), est dans les toutes meilleures équipes du monde dans les deux formes du jeu, test cricket et One-day International. 

\fig{images/cricket}{Un lanceur et les deux batteurs lors d'un match de cricket.}{fig:cricket}
\vspace{-8mm}

Bien que ce sport ne soit pas tr\`es populaire en France, il connait un v\'eritable succ\`es en \Nz. %qui accueillera, en association avec l'Australie, la Coupe du monde de cricket de 2015.
Notre colocataire, fan de cricket, a essay\'e de nous expliquer le principe, mais ce sport restera pour nous un sport o\`u les matchs sont vraiment longs et pr\'esentent donc tr\`es peu d'int\'er\^et \`a regarder \`a la t\'elevision. cependant la pratique semble plut\^ot ludique.

%\newpage
% --------------------------------------------------------------------

\section{\'Ecologie}
\label{sub:eco}

\subsection{\'Energie}
\label{sub:power}

Malgré une population relativement faible et d'abondantes ressources naturelles, la \Nz est un importateur net d'énergie. 
L'importation d'énergie est principalemt sous forme de produits pétroliers. 
Environ 35\% d'énergie primaire provient de sources d'énergie renouvelable (c.f. Tab.~\ref{tab:energy}). 

Les \Nzs consomment un peu moins que les autres pays de l'OCDE\footnote{Organisation {\tiny de} Coopération {\tiny et de} Développement \'Economiques} avec une consommation d'énergie de 4,53~TEP\footnote{Tonne d'\'Equivalent Pétrole} par habitant, conter une moyenne de 4,67 TEP dans l'OCDE. 

La \Nz est l'un des 13 pays de l'OCDE qui ne poss\`edent aucune centrale nucléaire. Ceci s'explique \`a cause des risques sismiques tr\`es importants dans la r\'egion et la manque de confiance des \Nzs dans l'\'energie atomique.

\begin{table}[h]
  \begin{center}
    \begin{footnotesize}
      \begin{tabular}{|c|c|c|c|c|}
        
        \hline
        P\'etrole & \'Energie Propre & Gaz & Charbon & G\'eothermie \\
        \hline
        35.70\% & 35.24\% & 20.73\% & 8.12\% & 0.19\%\\
        \hline
      \end{tabular}
      
    \end{footnotesize}
    \caption{Production d'\'energie en \Nz.}
    \label{tab:energy}
  \end{center}
\end{table}
\vspace{-5mm}

Les collines proches de Palmerston North -- \emph{Tatatua} et \emph{Ruahine Ranges} -- h\'ebergent un des plus grand parc \'eolien de l'h\'emisph\`ere sud avec plus de 194 turbines. Elles garantissent de l'\'electricit\'e \`a environ la moitier de la ville, soit 40.000 personnes.


%Pas de nucl\'eaire, beaucoup d'\'eolien.

%% \subsection{D\'echets - Gapillages}
%% \label{sub:waste}

%% La \Nz est confront\'e aux m\^emes probl\`emes que les pays europ\'eens au niveau des d\'echets. A cause d'un sur-emballage constant des produits quotidients, les \Nzs atteignent rapidement leurs quotas de d\'echets m\'enag\'es. Il leur faut payer davantage \`a la commune afin de pouvoir avoir plus de poubelles.
%% Ceci incite une partie des \Nzs \`a ne pas jeter correctement leurs ordures que l'on retrouvera apr\`es dans les rues ou le long des routes de campagne.

%% Le recyclage des d\'echets est assez r\'ecent, pour l'instant seul le verre et les cannettes sont tri\'es puis recycl\'es.

%% Le syst\`eme d'eau courante est gratuit, ceci permet un acces \`a tous \`a l'eau potable, mais en contrepartie la population fait moins attention au gaspillage.

\subsection{Pollution}
\label{sub:pollution}

Les villes \'etant tr\`es \'etal\'ees dans l'espace, la pollution est peu resentie mais elle est n\'eanmoins pr\'esente.
Le commerce important de v\'ehicules d'occasions implique un grand nombre de voitures tr\`es polluantes, avec une consommation d'essence excessive.

Les voitures ont le droit de circuler sur certaines plages. En plus de la pollution visuelle, olfactive et sonore \`a cause des nombreux 4$\times$4, les gaz d'\'echappement colorent le sable d'un gris fonc\'e d\'etruisant peu \`a peu les plages \nzes.

Une pollution des rivi\`eres est \'egalement constat\'ee \`a cause des usines install\'ees le long des principaux cours d'eau. 
Ceci repr\'esente un vrai probl\`eme pour les \'eleveurs qui utilisent l'eau des rivi\`eres pour abreuver les troupeaux de v\^aches ou de moutons.

Une catastrophe \'ecologique est survenue durant le mois d'octobre 2011.
Le \emph{Rena}, un porte-conteneurs, s'est \'echou\'e le long des c\^otes dans la \emph{Bay of Plenty}.
Les premi\`eres nappes de p\'etrole sont venues souiller les plages d'un petit paradis touristique qui est d\'esormais interdit au public.

%\newpage
% --------------------------------------------------------------------

\section{Paysages}
\label{sec:landscape}

%Fabuleux paysages, grands espaces verts pour les moutons/vaches. Tr\`es peu de villes. doc\footnote{\tt{http://www.doc.govt.nz/}}.

Avant l'arriv\'ee de l'homme, la \Nz \'etait recouverte \`a 85\% par des for\^ets.
Maintenant ce sont des plaines verdoyantes qui s'\'etendent \`a perte de vue afin de laisser pa\^{\i}tre les v\^aches et les moutons. 
Pour rappel, il y a environ 10 fois plus de moutons que de \Nzs en \Nz.


Afin de pr\'eserver au mieux l'environnement, un d\'epartement du gouvernement s'occupe de la conservation de l'h\'eritage naturel et historique du pays.
Il s'agit du \emph{Department of Conservation}, appel\'e DOC\footnote{\tt{http://www.doc.govt.nz/}}.
Ce d\'epartement s'occupe \'egalement de l'entretient des diff\'erents sentiers de randonn\'ees ainsi que celui des \emph{Huts}, qui sont des refuges o\`u l'on peut trouver un lit et du chauffage moyennant une dizaine d'euros.


\subsection{Visites en Van}
\label{sub:van}

Le meilleur moyen pour visiter la \Nz est la location/achat d'un \emph{van} ou d'un camping-car.
Il existe de nombreux \emph{backpackers}\footnote{Sorte d'auberge ou l'on peut dormir, se doucher et se faire \`a manger pour quelques dizaines de dollars (autour de 20\euro)}, qui proposent souvent la location de \emph{vans} ainsi que des assurances pour permettre \`a tout le monde de partir facilement dans des coins recul\'es du pays. 

\fig{images/van}{D\'ecouverte de la \Nz en camping-car}{fig:van} 

La plupart des points de vue ne sont pas accessibles en transport en commun car il faut toujours s'\'eloigner des grandes villes en voiture, puis s'\'eloigner des routes \`a pied.
Il est donc impensable d'organiser une visite de la \Nz sans moyen de locomotion personnel.

Bien que l'achat puis la revente d'une voiture r\'eduit fortement le co\^ut du voyage, il est parfois pr\'ef\'erable de se d\'eplacer en avion, puis de louer une voiture une fois \`a destination.
Les prix propos\'es par les compagnies a\'eriennes \emph{low-cost} permettent cette pratique lorsqu'on se d\'eplace seul.

Un compromis lors des voyages \`a deux est l'achat un \emph{break}, mois cher et consommant moins qu'un \emph{van}, mais offrant tout de m\^eme la possibilit\'e de dormir \`a l'arri\`ere. C'est cette solution que nous avons retenu pour notre voyage \`a travers la \Nz, afin d'acc\'eder aux endroits inacessibles comme les c\^otes de \emph{White Cliff} (figure~\ref{fig:cliff}).
\fig{images/cliff}{Falaises de \emph{White Cliff}}{fig:cliff}

\subsection{Le Seigneur des Anneaux}
\label{sub:lotr}

La \Nz a servi de d\'ecor pour le film \textit{Le Seigneur des Anneaux}, r\'ealis\'e par Peter Jackson. 
Il a su exploiter au maximum la beaut\'e de chacun des lieux de tournage afin d'obtenir un rendu final s'approchant du domaine du fantastique.
%r\'eussi \`a mettre en valeur les principales r\'egions \`a visiter. 

Les pleines de \textit{Matamata} ont acceuilli la \textit{Comté} des \textit{hobbits} (voir figure~\ref{fig:hobbit}). Bien qu'elles soient moins bien d\'ecor\'ees que dans le films, on peut visiter ces maisons atypiques. 

Pour ceux qui n'ont pas peur de marcher, on peut suivre les traces de \emph{Frodon} en escaladant le \textit{Mordor} qui est en r\'ealit\'e le mont \textit{Tongariro}. 

\smallfig{0.5}{images/hobbit}{Reste de la \emph{Comté} apr\`es le tournage.}{fig:hobbit}

Les \Nzs sont fiers de cette publicit\'e et proposent d\'esormais des \emph{tour-operator} afin de repasser sur les lieux des tournages.
Depuis la sortie de la trilogie, de plus en plus de voyageurs choisissent la \Nz comme destination pour des vacances, des voyages de noces ou pour leurs \'etudes. 

%\newpage
% --------------------------------------------------------------------

\section{Conclusion}
\label{sec:conclusion}


J'ai vraiment appr\'eci\'e cette d\'ecouverte de la \Nz, tant au niveau de la richesse des paysages que de la diversit\'e de la flore.

Cependant certains aspects viennent noircir ce petit paradis.
Le mode de vie des \Nzs ressemble beaucoup au mode de vie occidental donc il n'y a pas eu le d\'epaysement que l'on pouvait attendre en partant si loin.
La disparition progressive de la culture M\=aori diminue consid\'erablement la richesse culturelle du pays.
Un des buts de ce voyage, s'inpr\'egner d'une nouvelle culture, n'a donc pas \'et\'e atteint.

D'autre part, j'ai \'et\'e un peu choqu\'e par le retard technologique aussi bien dans la vie courante (t\'el\'ephones portables, tablettes tactiles, internet) que dans le domaine universitaire (machines outils, robots pour la recherche).
%Le manque d'animation rend la vie en \Nz assez ennuyante. 
%D'autre part la v\'etust\'e des infrastructures universitaires ne permet pas d'\'etudier correctement.
%\textbf{La culture Maori disparait peu \`a peu.}

La \Nz restera tout de m\^eme un tr\`es bon souvenir duquel nous rapporterons de tr\`es nombreuses photos.

\section*{Remerciements}

Je tiens \`a remercier M. Bouzgarrou qui m'a offert l'opportunit\'e d'effectuer mon stage en \Nz. 
Je remercie \'egalement M. Flemmer qui m'a accueilli au sein du laboratoire de robotique de Massey.

Je ne peux pas oublier de mentionner les efforts et la patience n\'ecessaire \`a nos colocataires \nzs afin comprendre notre anglais approximatif.

Je remercie \'egalement le groupe de stagiaires fran\c{c}ais avec qui j'ai pu discuter pour ne pas me sentir perdu au travail \`a l'autre bout du monde.

%\newpage
%Commentaires :
%\end{cursive}
\end{document}


