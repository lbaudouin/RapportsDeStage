\documentclass[a4paper,oneside,12pt]{article}

\usepackage[utf8]{inputenc} %Prise en charge des accents
\usepackage{csquotes}
\usepackage[english, french]{babel} %Charge les langues
\usepackage{color} %Charge les couleurs
\usepackage{fancyhdr} %Utilisation des entêtes
\usepackage{lastpage} %Script pour avoir le nombre de pages
\usepackage{graphicx} %Permet d'inclure des images
%\usepackage{makeidx} 
\usepackage{amsmath}
\usepackage{xspace} %Espaces intelligents
\usepackage{caption}
\usepackage{subfig}
\usepackage{multirow}
\usepackage{tabularx}
%\usepackage{picins} %Permet de faire des sous-figures
\usepackage{tikz} %Permet de dessiner
\usepackage[backend=bibtex]{biblatex}
\addbibresource{papers.bib}

%Fonctions personnelles
\newcommand{\saut}{\vspace{5mm}}
\newcommand{\eme}[1]{$#1^{\grave{e}me}$}
\newcommand{\ere}[1]{$#1^{\grave{e}re}$}
\newcommand{\er}[1]{$#1^{er}$}
\newcommand{\red}[1]{\textcolor{red}{#1}}
\newcommand{\warning}[1]{\textbf{\textcolor{red}{#1}}}

\newcommand{\fig}[3]{
  \begin{figure}[ht]
    \begin{center}
      \includegraphics[width=0.8\linewidth]{#1}
      \caption{#2}
      \label{#3}
    \end{center}
  \end{figure}
}

\newcommand{\smallfig}[4]{
  \begin{figure}[ht]
    \begin{center}
      \includegraphics[width=#1\linewidth]{#2}
      \caption{#3}
      \label{#4}
    \end{center}
  \end{figure}
}

%Noms
\newcommand{\leo}{L\'eo \textsc{Baudouin}\xspace}

%Logos
\newcommand{\logoIFMA}{images/logo-IFMA.jpg}
\newcommand{\logoIP}{images/logo-IP.jpg}
\newcommand{\logoLASMEA}{images/logo-LASMEA.png}
\newcommand{\logo}{images/videH.jpg}

%Informations
\newcommand{\titre}{Rapport de Projet de fin d'\'Etude}
\newcommand{\sujet}{Cartographie, localisation et navigation muti-robots utilisant la vision omnidirectionnelle}%{Multi-Robots mapping/localization and navigation using omnidirectional vision}
\newcommand{\auteur}{\leo}
\newcommand{\pole}{MMS -- M\'ecatronique}
\newcommand{\annee}{\eme{4} \textit{année -- P\^ole MMS -- Mécatronique}}
\newcommand{\dates}{du 01 mars 2012 au 29 février 2015}
\newcommand{\lieu}{Institut Pascal (LASMEA), France}
\newcommand{\adresse}{Institut Pascal (LASMEA)\\
  \begin{footnotesize}
    UMR 6602 UBP / CNRS\\
    Campus des Cézeaux\\
    24, Avenue des Landais\\
    63177 \textsc{Aubière} Cedex, \textsc{France} 
  \end{footnotesize}
}
\newcommand{\tuteur}{Youcef \textsc{Mezouar}}
\newcommand{\tuteurIFMA}{Belhassen-Chedli \textsc{Bouzgarrou}}

%Identification du document
\newcommand{\file}{RapportPIFE-lbaudouin.pdf}
\newcommand{\resumeFR}{Ce projet consiste à mettre au point des techniques de cartographies à l'aide d'une flotte de robots. La particularité consiste à utiliser des types de caméras différents sur les robots. Les robots doivent créer une carte locale de l'environnement qu'ils ont parcouru. Un algorithme de fusion de carte vient ensuite concaténer toutes les informations obtenues dans une carte globale. Le but final est de permettre à une flotte de robots de naviguer en autonomie dans un espace donné.}
\newcommand{\resumeEN}{This project plan to create mapping algorithms with a fleet of robots. The main improvement is to use several type of cameras on robots. Robots have to create a local map from environment. A map merging algorithm will then concat all informations from local maps. The main issue is to allow to a fleet of robots to navigate autonomously in given environment.}
\newcommand{\motsclefsFR}{Robotique, navigation autonome, fusion de carte}
\newcommand{\motsclefsEN}{Robotics, autonomous navigation, map merging}


%Défninition des longueurs
%\setlength{\topmargin}{-1.cm}  
\setlength{\textheight}{23.5cm}    
\setlength{\oddsidemargin}{0.cm}
\setlength{\evensidemargin}{0.cm}
\setlength{\textwidth}{16.cm}

%Début du document
\begin{document}

%Page de garde

\thispagestyle{empty}
\textcolor{blue}{Ministère de l'Enseignement Supérieur et de la Recherche}\\
\vspace{5mm}
\begin{centering}

  \includegraphics[height=2.0cm]{\logoIFMA} ~
  \includegraphics[height=2.0cm]{\logoLASMEA} ~
  \includegraphics[height=2.0cm]{\logoIP}

\end{centering}

\vspace{5mm}
\begin{center} 
\begin{large}
\textbf{\titre}
\end{large}
\\\vspace{3mm}--\vspace{3mm}\\
\begin{Huge}
\textbf{\sujet}
\end{Huge}
\\
\vspace{10mm}
\begin{LARGE}
\auteur\\

\vspace{5mm}
\annee\\
\end{LARGE}
\vspace{5mm}
\today
\end{center}
%\vspace{5mm}

\begin{flushright}
	\begin{large}
		Laboratoire :\\
		
		%\vspace{2mm}
		\textbf{\lieu}\\
		
		\vspace{3mm}
		Tuteur :\\
		
		%\vspace{2mm}
		\textbf{\tuteur}\\
		
		\vspace{3mm}
		Tuteur IFMA :\\
		
		%\vspace{4mm}
		\textbf{\tuteurIFMA}\\
		
		\vspace{3mm}
		Date : \dates
	\end{large}
\end{flushright}

\vspace{5mm}

\begin{tabular}{l r}
	\begin{minipage}{0.3\linewidth}
		\includegraphics[height=2cm]{images/EU.jpg}\\
		\includegraphics[height=2cm]{images/auvergne.png}
	\end{minipage}
	&
	\begin{minipage}{0.7\linewidth}
		\begin{flushright}
			\adresse
			\vspace{2mm}
			
			\textcolor{red}{I\textsc{nstitut} F\textsc{ran\c{c}ais de} M\textsc{\'ecanique} A\textsc{vanc\'ee}}\\
			\begin{footnotesize}
				\textcolor{blue}{C\textsc{ampus de} C\textsc{lermont}-F\textsc{errand} - L\textsc{es} C\textsc{\'ezeaux} - \textsc{bp} 265\\
				63175 A\textsc{ubi\`ere} C\textsc{edex} - F\textsc{rance}}\\
				\textcolor{blue}{T\textsc{el}. +33 (0)4 73 28 80 00 - F\textsc{ax} +33 (0)4 73 28 81 00\\
				leo.baudouin@ifma.fr -} \textcolor{red}{www.ifma.fr}\\
			\end{footnotesize}
		\end{flushright}
	\end{minipage}
\end{tabular}



\newpage

\setlength{\topmargin}{-1.cm}  

\thispagestyle{empty}

%\addcontentsline{toc}{section}{\numberline{}Identification}

\noindent \textcolor{blue}{Minist\`ere de l'Enseignement Sup\'erieur et de la Recherche}

\begin{flushright}
\vspace*{-0.80cm}
\includegraphics [width=3.0cm]{\logoIFMA}
\end{flushright}
\vspace{-1cm}
%%% Fin En-t�te
%----------------------------------------------------------------------------------------
%%% Premier cadre : Identification
\vfill
\rule[6pt]{16cm}{1pt}
\large{TITRE DU RAPPORT : \\
\textbf {\titre}
\newline
\large{\hfill \title \\}
\rule[6pt]{16cm}{.5pt}
\large{AUTEUR(S) :  {\auteur \hfill \pole}\\}
\rule[0pt]{16cm}{.5pt}
\begin{center}
\begin{tabular}{ccc}
Date du document & Nb pages & R\'ef\'erence du document \\ 
\today & \pageref{LastPage} & \file  %
\end{tabular}
\end{center}
\rule[6pt]{16cm}{1pt}
%----------------------------------------------------------------------------------------
\vfill
\noindent \rule[0pt]{16cm}{1pt}
\large{R\'ESUM\'E :} \\ \normalsize {\resumeFR} \\
\rule[0pt]{16cm}{.5pt}
\large{Mots-cl\'e : \motsclefsFR \\}
\rule[0pt]{16cm}{1pt}
%----------------------------------------------------------------------------------------
\selectlanguage{english}
\vfill
\label{English}
\noindent\rule[0pt]{16cm}{1pt}
\large{ABSTRACT:} \\ \normalsize {\resumeEN} \\
\rule[0pt]{16cm}{.5pt}
\large{KeyWords: \motsclefsEN \\}
\rule[0pt]{16cm}{1pt}
%----------------------------------------------------------------------------------------
\selectlanguage{francais}%
\vfill

\setlength{\topmargin}{0.cm}

\newpage

%Style des pages
\pagestyle{empty}
\fancyhf{} %Entête automatique désactivée (mettre en commentaire pour l'activer)
\fancyhead[L]{Rapport de Projet de fin d'\'Etude}
\fancyfoot[R]{\thepage/\pageref{LastPage}}
\fancyfoot[L]{\leo~-~\annee}

%Sommaire
\pagestyle{fancy}
\tableofcontents
\addtocontents{toc}{\protect\thispagestyle{fancy}}
\addtocontents{toc}{\protect\pagestyle{fancy}}
\newpage

\section*{Avant Propos}
\addcontentsline{toc}{section}{\numberline{}Avant-propos}

Je tiens tout d'abord à remercier l'IFMA d'avoir accepté que le début de ma thèse soit compté comme mon stage de fin d'étude (PIFE).
Effectivement suite à un master, réalisé en parallèle des enseignements de l'IFMA, j'ai souhaité poursuivre dans la recherche au travers d'un doctorat en robotique mobile.

\'Etant donné que les premiers mois d'une thèse sont principalement destinés à faire une étude de l'état de l'art dans le domaine concerné, ce rapport contiendra un nombre plus ou moins important de citations d'articles scientifiques et moins de résultats concrets. Il sera donc sensiblement différent d'un rapport de PIFE traditionnel.

Ayant déjà travaillé sur plusieurs projets dans la robotique mobile lors des stages proposés par l'IFMA, et avec un master en \emph{Image \& Vision} en poche, j'ai décidé de travailler sur un sujet cumulant ces deux aspects scientifiques, la vision embarquée pour la robotique mobile. Je suis donc passé des robots humano\"{\i}des aux robots roulants.

Je remercie \tuteur, mon tuteur de thèse, de m'avoir accepté dans son équipe de recherche. Je remercie également Omar \textsc{Ait-Ader} et Jonathan \textsc{Courbon} pour l'aide qu'ils m'ont apporté pendant le début de cette thèse.
\newpage

\section{Introduction}

Avec les GPS, les radars de recul, les assistants de créneaux et autres gadgets, les aides à la conduite deviennent de plus en plus intelligentes.
Devant le nombre de capteurs et de calculateurs embarqués on est en droit de se demander si le conducteur humain est encore nécessaire dans une voiture.

Il existe déjà des projets comme la voiture Google qui roule de façon autonome sur certaines routes américaines, ou les voitures Volvo qui peuvent suivre un camion sous forme de convoi de quelques véhicules.
Mais ces voitures nécessite un nombre très important de capteurs avec une précision poussée.
On y retrouve au choix : un GPS précis au centimètre, des télé-mètres lasers plans ou 3D, ainsi que des accéléromètres multi-directionnels le tout souvent complété par quelques caméras embarquées.

Tous ces capteurs de précision ont un coût important et ne sont pas à la porté des particuliers.
C'est pourquoi nous travaillons sur des véhicules autonomes utilisant la vision, le prix d'une caméra étant largement inférieur à celui d'un télémètre laser 3D.

\smallfig{0.4}{images/VIPALAB.jpg}{Un VipaLab, véhicule sans pilote présent à l'Institut Pascal}{fig:vipalab}

Cependant, les algorithmes en cours de développement ne sont pas encore adaptés à des véhicules roulant à haute vitesse.
En attendant que le matériel et les programmes le permettent, nous allons utiliser un véhicule autonome roulant à basse vitesse, un VipaLab (voir figure~\ref{fig:vipalab}).


Le but final est d'avoir une flotte de véhicules intelligents, en libre service, dans le centre ville de Clermont-Ferrand.
A partir d'un point initial, le robot devra être capable de planifier son trajet jusqu'à la destination souhaitée par l'utilisateur et de l'exécuter de manière totalement autonome mais sécurisé.
Effectivement le robot pourra être amené à rouler dans des zones piétonnes (respect des piétons, évitement de collisions) mais également sur certaines portions de routes (respect du code de la route).

Dans un environnement aussi contraint qu'une zone piétonne, le robot doit pouvoir observer ce qu'il se passe tout autour de lui.
L'utilisation de caméras perspectives à l'avant et à l'arrière ne sera pas entièrement suffisant dans certains cas.
C'est pourquoi nous avons souhaité travailler avec des caméras omnidirectionnelles (caméra catadioptrique, voir figure~\ref{fig:camcata}).
Celles-ci permettent d'avoir un champs de vu à 360\degre.

\smallfig{0.6}{images/omni.png}{Caméra omnidirectionnelle composée d'un miroir parabolique et d'une caméra affine. L'image projetée est donc obtenue sous forme de disque.}{fig:camcata}

Notre flotte de véhicule ne sera pas obligatoirement homogène, on y retrouvera différents robots (véhicules) et différents capteurs (caméras).
La différences entre les robots n'impactera pas sur notre travail car la gestion de la trajectoire et de son suivi sera calculé par un programme tiers.
La différences entre les caméras aura, quand à elle, un impacte très important sur les travaux à réalisés, car la formation des images est différente ce qui implique des informations utilisables également différentes.
Il existe de nombreux algorithme de SLAM\footnote{Simultaneous Localisation And Mapping -- Cartographie et localisation simultanée} utilisant la vision.
Cependant ils ne se concentrent que sur un type de caméra à la fois.

L'innovation demandé dans ce travail de recherche est de mettre en place de nouveaux algorithmes permettant la cartographie et la navigation pour une flotte de robots munies de caméras de type différents.
Pour parler d'une paire de caméras composée d'une caméra perspective et d'une caméra omnidirectionnelle, nous utiliserons le terme de paire hybride, d'où la notion de vision hybride.

\vspace{5mm}
Après cette introduction, nous ferrons un point sur l'état de l'art des domaines concernés dans la section~\ref{sec:etatart}.
Il sera suivi, dans la section~\ref{sec:vision}, par les aspects mathématiques de la vision par ordinateur.
La section~\ref{sec:strategie} exposera la stratégie que nous suivrons pour résoudre le problème posé.
Nous terminerons par un point sur l'avancement du projet, avant une brève conclusion.

\newpage

\section{Section 1}
\subsection{Sous-Section 1}
\cite{Puig08}
\newpage

\section{Vision}
\label{sec:vision}

Cette partie sera consacrée à l'aspect théorique de la reconstruction 3D.
Dans un premier temps nous verrons le cas de la reconstruction à partir d'une paire de caméras classiques.
Nous verrons ensuite le modèle unifié pour les caméras catadioptriques.
%Nous aborderons au final la vision hybride.

\subsection{Reconstruction 3D}
\label{sub:reconstruction}

La reconstruction 3D à partir de deux vues perspectives a été très bien formalisée dans le livre de \citeauthor{Hartley03Book} \cite{Hartley03Book}.
En prenant deux vues en utilisant une même caméra perspective, nous devons calculer le déplacement entre les lieux de prise de vue.
Celui-ci est déterminé par une matrice de projection $\mathbf{P}$ qui se compose de deux matrices $\mathbf{R}$ et $\mathbf{t}$, respectivement la matrice de rotation et la matrice de translation.

La projection d'un point $\mathbf{X}$ de coordonnées $\begin{pmatrix}X&Y&Z\end{pmatrix}^\top$ dans l'espace donne le point $\mathbf{m}$ de coordonnées $\begin{pmatrix}u&v\end{pmatrix}^\top$ dans le plan image.
En utilisant les coordonnées homogènes, nous pouvons décrire un modèle linaire de projection dans le cas d'une caméra perspective :
\begin{equation}
s \underbrace{\begin{pmatrix}u \\ v \\ 1\end{pmatrix}}_{\mathbf{m}} = 
\underbrace{\begin{pmatrix}f_u && 0 && u_0 \\ 0 && f_v && v_0 \\ 0 && 0 && 1\end{pmatrix}}_{\mathbf{K}} . 
\underbrace{\begin{pmatrix}r11 && r12 && r13 && t1 \\ r21 && r22 && r23 && t2 \\ r31 && r32 && r33 && t3\end{pmatrix}}_{[\mathbf{R~t}]} . 
\underbrace{\begin{pmatrix}X \\ Y \\ Z \\ 1\end{pmatrix}}_{\mathbf{X}}
\label{eq:projection}
\end{equation}

\begin{equation}
s.\mathbf{m} = \mathbf{K}.\mathbf{P}.\mathbf{X}
\end{equation}
Avec : 
\begin{itemize}
\item $\mathbf{m}$ le point dans l'image
\item $\mathbf{P}$ la matrice de projection 
$\mathbf{P} = [ \mathbf{R} ~ \mathbf{t} ]$
\item $\mathbf{K}$ la matrice des paramètres intrinsèques (calibration)
\item $s$ est un facteur réel quelconque ($s \in \Re$)
\end{itemize}

Ceci est un modèle simplifié ne prenant pas en compte les différentes distorsions que l'on peut retrouver dans une image (distorsions radiales, distorsion tangentielle, \dots).

\subsubsection{Calibration}
\label{subsub:calibration}
La calibration d'une caméra perspective est réalisée à partir d'une mire composée généralement par une grille ou de points \cite{HoraudBook, Hartley03Book}.

L'opération de calibration de caméra revient à modéliser le processus de formation des images, c'est-à-dire trouver la relation entre les coordonnées spatiales d'un point de l'espace et le point associé dans l'image prise par la caméra.

Dans le cas d'une caméra perspective, le modèle de projection est linéaire, comme présenté dans l'équation~\ref{eq:projection}.
La matrice de projection ou matrice de paramètres extrinsèques $\mathbf{P}$, dépend de la position de la caméra et non du capteur lui-même.
La calibration consiste donc à estimer la matrice des paramètres intrinsèques $\mathbf{K}$ ainsi que les éventuelles distorsions de l'image.
$$
\mathbf{K} = \begin{pmatrix}f_u && 0 && u_0 \\ 0 && f_v && v_0 \\ 0 && 0 && 1\end{pmatrix} 
$$
Avec $f_u$ et $f_v$ les coefficients dépendants des focales selon les axes $\vec{x}$ et $\vec{y}$ de l'image.
Le point de coordonnées $\begin{pmatrix}u_0&v_0\end{pmatrix}^\top$ est le centre de l'image issue du capteur optique.

\subsubsection{Méthode}
\label{subsub:reconstruction}

Comme nous avons pu voir précédemment, les seules informations que nous allons utiliser dans les images sont des points.
Pour donner un sens à ces listes de points, ils seront appairés.
C'est à dire que chaque point dans la première image devra avoir un point correspondant dans la seconde.

On va pouvoir calculer une matrice regroupant toutes les informations possible de tirer de cette configuration.
Cette matrice se nommera la matrice fondamentale, notée $\mathbf{F}$.
Elle est définie pour toutes paires de points $i$ dans les images 1 et 2 (par la suite l'indice $i$ sera omis) comme solution de l'équation :
\begin{equation}
\mathbf{m}_{i,1}^ \top.\mathbf{F}.\mathbf{m}_{i,2} = 0
\label{eq:fondamentale}
\end{equation}
Connaissant la matrice intrinsèque de la caméra (voir \ref{subsub:calibration}), nous pouvons éliminer les facteurs induits par les focales, afin de normaliser la matrice fondamentale.
On obtiendra la matrice essentielle $\mathbf{E}$ :
\begin{equation}
\mathbf{E} = \mathbf{K}_1^{\top} . \mathbf{F} . \mathbf{K}_2
\end{equation}
Si les deux images sont acquises par la même caméra, on aura $\mathbf{K}_1 = \mathbf{K}_2$.
Nous pouvons obtenir la matrice essentielle directement depuis la liste de points, en normalisant les points eux-même.
\begin{equation}
(\mathbf{m}_{i,1}.\mathbf{K}_1^{-1})^\top.\mathbf{E}.(\mathbf{K}_2^{-1}.\mathbf{m}_{i,2}) = 0
\label{eq:essentielle}
\end{equation}


Afin d'obtenir les matrices $\mathbf{R}$ et $\mathbf{t}$, nous devons utiliser une autre formulation de la matrice essentielle $\mathbf{E}$ :
\begin{equation}
\mathbf{E} = \mathbf{R} . [\mathbf{t}]_\times
\end{equation}
Avec $[\mathbf{t}]_\times$ la matrice anti-symétrique\footnote{Si $\mathbf{t}=\begin{pmatrix}t_1\\t_2\\t_3\end{pmatrix}$, alors $[\mathbf{t}]_\times=\begin{pmatrix}0&-t_3&t_2\\t_3&0&-t_1\\-t_2&t_1&0\end{pmatrix}$} du vecteur $\mathbf{t}$.
Nous pouvons maintenant extraire $\mathbf{R}$ et $\mathbf{t}$. Pour cela nous devons décomposer la matrice essentielle  à l'aide d'une décomposition en valeurs singulières :
$$\mathbf{E}=\mathbf{U} \mathbf{\Sigma} \mathbf{V}^{\top}$$
Comme expliquer dans~\cite{Hartley03Book}, la matrice essentielle est de rang 2 et les deux valeurs singulières sont égales.
Nous avons donc : $\mathbf{\Sigma} = \begin{pmatrix}s&0&0\\0&s&0\\0&0&0\end{pmatrix}$.\\
Alors, en définissant   $\mathbf{W}=\begin{pmatrix}0&1&0\\-1&0&0\\0&0&1\end{pmatrix}$ on obtient :
\begin{equation}
[\mathbf{t}]_\times = \mathbf{V} \mathbf{W} \mathbf{\Sigma} \mathbf{V}^{\top}
\end{equation}
\begin{equation}
\mathbf{R} = \mathbf{U} \mathbf{W}^{-1} \mathbf{V}^{\top}
\end{equation}
Nous avons donc obtenu la position relative (rotation et translation) de la caméra à partir de deux listes de points issues d'images perspectives.
La triangulation peut donc avoir lieu.


\subsection{Modèle unifié}

Pour les caméras catadioptriques, un modèle permettant de représenter les caméras munies d'un miroir plan, parabolique, hyperbolique, elliptique a été mis en place.

Ce modèle unifié, appelé modèle sphérique, prend en entrée un seul paramètre $\xi$ pour représenter le miroir.
Une série de projection va venir modéliser le couple miroir-caméra.

\smallfig{0.6}{images/modeleunifie.png}{Modèle sphérique pour caméras catadioptriques comme défini dans \cite{Courbon09PhD}}{fig:modeleunifie}

\subsubsection{Modèle de projection}

Soit $\mathbf{X}$ un point dans le repère monde.
On suppose la caméra au point $O$ de coordonnées $(0,0,0)$ avec l'axe $\vec{z}$, l'axe principal de la caméra.

On va projeter le point $X$ sur la sphère unitaire centrée en $O$. On obtiendra le point $X_m$ :
\begin{equation}
X_m = \frac{X}{\rho}
\end{equation}
Avec $\rho = ||X|| = \sqrt{X^2+Y^2+Z^2}$.

On projette enfin le point $X_m$ sur le plan de $Z = 1 - \xi$ :
\begin{equation}
x = \begin{bmatrix}  \frac{X}{Z+\xi \rho} && \frac{Y}{Z +\xi \rho} && 1 \end{bmatrix}^{\top}
\end{equation}

Pour finir on applique la transformée homographique due aux paramètres intrinsèques de la caméra, pour obtenir le point $m$ dans l'image.
\begin{equation}
m= K x
\end{equation}
L'ensemble de la projection n'est pas linéaire et ne peut donc pas s'écrire sous la forme une équation comme pour le modèle perspectif.

\subsubsection{Rétro-projection}

A partir du point $m$ dans l'image on souhaite avoir le point 3D dans le repère monde correspondant.

\begin{equation}
x = \begin{bmatrix} x && y && 1 \end{bmatrix}^{\top} = K^{-1} m
\end{equation}

On inverse la fonction de projection pour obtenir les coordonnées du point de la shpère :
\begin{equation}
X_m = (\nu^{-1} + \xi) \bar{x}
\end{equation}
\begin{equation}
\bar{x} = \begin{bmatrix}x && y && \frac{1}{1+\xi \mu} \end{bmatrix}^{\top}
\end{equation}
Avec :
\begin{equation}
  \left \{
  \begin{matrix}
    \mu = \frac{-\gamma-\xi(x^2+y^2)}{\xi(x^2+y^2)-1} \\
    \gamma = \sqrt{1+(1-\xi^2)(x^2+y^2)}
  \end{matrix}
 \right.
\end{equation}
Nous avons maintenant un point sur la sphère correspondant au point image.
Sachant que lors de la projection celui-ci est sur la droite reliant le point 3D et le centre de la sphère, nous pouvons conclure que nous connaissons la direction du point 3D réel, puisqu'il est porté par la droite $(OX_m)$.
En ayant deux points de vues distincts, il suffit de trouver le croisement des deux droites pour reconstruire le point 3D.

\subsubsection{Déplacement}

Dans sa thèse \cite{Puig11PhD}, \citeauthor{Puig11PhD} expose la partie mathématique pour calculer le déplacement de la caméra entre deux prises d'images omnidirectionnelles.
Ceci est donc juste un rappel d'une méthode existante, analogue à la version perspective.
On notera avec un accent circonflexe les vecteurs ou matrice qui seront \emph{liftés}.
Soit $q=\begin{pmatrix}q_1&q_2&q_3&q_4\end{pmatrix}^\top$, on aura $\hat{q} = \begin{pmatrix}q_1^2&q_1q_2&q_2^2&q_1q_3&q_2q_3&q_3^2\end{pmatrix}^\top$

On définit ensuite une relation entre les points 3D les points 2D :
\begin{equation}
\widehat{[q]}_\times \text{P}_{cata} \hat{Q} = 0
\end{equation}
Que nous pouvons mettre sous forme d'un système d'équations $6n\times60$:
\begin{equation}
\left( \hat{Q}^{\top} \otimes \widehat{[q]}_\times \right) \text{P}_{cata}  = 0_6
\end{equation}
($\otimes$ est le produit de Kronecker)

On peut résoudre ce système d'équations par la méthode des moindres carrés en utilisant la SVD\footnote{Singular Value Decomposition}.
On remarque que le rang d'une matrice symétrique $3\times3$ est 2, et que sa version \emph{liftée} est de rang 3.
Il faudra donc avoir au minimum 20 points pour résoudre le système car ils n'apportent que 3 équations indépendantes par paires.

%\begin{equation}
%c_x = \frac{M_{46}}{M_{66}}
%\end{equation}
%\begin{equation}
%c_y = \frac{M_{56}}{M_{66}}
%\end{equation}
%\begin{equation}
%\xi = \sqrt{\frac{\frac{M_{16}}{M_{66}}-c_x^2}{-2\left( \frac{M_{44}}{M_{66}}-c_x^2 \right)}}
%\end{equation}
%\begin{equation}
%f = \sqrt{2\left(2\xi^4 + \left(1-\xi^2\right)^2\right)\left(\frac{M44}{M66}-c_x^2\right)}
%\end{equation}

On définit une matrice de transformation $\text{T}_{cata}$ (voir \cite{Puig11PhD} pour la définition de $X_\xi$) :
\begin{equation}
\text{T}_{cata} = R_{6\times6} \left(I_6~T_{6\times4} \right) \sim (\hat{K}X_\xi )^{-1} \text{P}_{cata}
\end{equation}

Pour obtenir $T_{6\times4}$ il suffit de multiplier $\text{T}_{cata}$ par l'inverse de $\hat{R}_{est}$.
Il nous reste maintenant à extraire la matrice de rotation :
\begin{equation}
\hat{R} = \hat{R}_z(\gamma).\hat{R}_y(\beta).\hat{R}_x(\alpha)
\end{equation}
$\hat{R}_{est}$ est la matrice $6\times6$ la plus à gauche de $\text{P}_{cata}$.
\begin{equation}
\gamma = \tan^{-1}\left(\frac{\hat{R}_{est,51}}{\hat{R}_{est,41}}\right)
\end{equation}
$\hat{R}_{est} = \hat{R}_z^{-1}(\gamma).\hat{R}_{est}$
\begin{equation}
\beta = \tan^{-1}\left(\frac{-\hat{R}_{est,52}}{\hat{R}_{est,22}}\right)
\end{equation}
$\hat{R}_{est} = \hat{R}_y^{-1}(\beta).\hat{R}_{est}$
\begin{equation}
\alpha = \tan^{-1}\left(\frac{\hat{R}_{est,42}}{\hat{R}_{est,22}}\right)
\end{equation}
Une fois les trois angles obtenus, nous pouvons reconstruire la matrice de rotation classique $\mathbf{R}$.

Cependant, lors des simulations, les résultats obtenus pour le déplacement de la caméra étaient tous faux.
Lors d'un déplacement en ligne droite, l'estimation du mouvement et la reconstruction 3D sont corrects, mais dès qu'une rotation est amorcée l'algorithme ne converge plus vers la solution réelle.

%\subsection{Vision Hybride}
%Puig \cite{Puig08}

%Pour les points de l'image catadioptrique, nous devons utilisé les coordonnées \cite{lifted}:
%$$\hat{\mathbf{q}}=\begin{pmatrix}q_1^2+q_2^2\\q_1q_2\\q_2q_3\\q_3^2\end{pmatrix}$$
%Nous avons :
%\begin{equation}
%\hat{\mathbf{q}}_c^{\top}\mathbf{F}_{cp}\mathbf{q}_p=0
%\end{equation}
%On défini $\mathbf{B}_c$ ($3\times4$ matrice) pour représenter la matrice de projection pour la caméra catadioptrique.
%La matrice essentielle ne peut pas \^etre calculé comme dans la section~\ref{sub:reconstruction} car $\mathbf{B}_c$ n'est pas une matrice de rotation :
%\begin{equation}
%\mathbf{E} \neq \mathbf{B}_c^{\top} \mathbf{F}_{cp} \mathbf{K}_p
%\end{equation}

\newpage

\section{Expériences}
\label{sec:exp}

\subsection{Environnement}
\label{sub:env}

Afin de pouvoir éviter les obstacles le robot a besoin de connaître précisément où ils se situent autour de lui. Dans un premier temps nous devions utiliser la vision afin d'obtenir leurs coordonnées, cependant ce projet a pris un peu de retard. 
J'ai donc du utiliser un système de repérage global, comme le système de \emph{motion capture}, présent dans la salle d'expérience de robotique du LAAS.

L'utilisation est assez simple : à l'aide de 10 caméras infrarouges, le logiciel Cortex2.0 retrouve la position 3D des \emph{markers} (voir fig.\ref{fig:marker}) apposés sur les différents objets. 
Lors de la création du modèle de l'objet (voir fig.\ref{fig:objet}), les longueurs entre les \emph{markers} sont sauvegardées.


\begin{figure}[h]
\begin{center}
\includegraphics[width=8.0cm]{images/marker_small.jpg}
\caption{Un \emph{marker} réfléchissant les infrarouges pour la localisation.}
\label{fig:marker}
\end{center}
\end{figure}

\begin{figure}[h]
\begin{center}
\includegraphics[width=7cm]{images/mocap.png}
\includegraphics[width=8.5cm]{images/mocap_real_small.jpg}
\caption{Utilisation de modèles dans Cortex2.0. Ici la table est localisée à l'aide de 4 \emph{markers} disposés de manière unique dans la scène.}
\label{fig:objet}
\end{center}
\end{figure}

Les positions des objets sont ensuite déduites toutes les $5ms$ en comparant les distances entre les \emph{markers} présents dans la scènes et les distances enregistrées dans les modèles. Une fois la position de l'objet calculée, les coordonnées sont \emph{streamées} sur le réseau afin de pouvoir être lues sur tous les ordinateurs du LAAS. On peut donc avoir en temps réel les informations (position, orientation) de chacun des objets présents dans la scène.

La précision de la \emph{motion capture} est de l'ordre du centimètre et la localisation d'un objet étant effectuée à l'aide de plusieurs points, la précision sur l'emplacement d'un objet est inférieur à un centimètre.

Notre scène contient les objets suivants \emph{trackés} par la \emph{motion capture} : 
\begin{itemize}
\item Une table roulante
\item Une chaise de bureau roulante
\item Une barre à $6cm$ de haut
\item Deux tores de $1m$ de diamètre
\item Un robot HRP-2
\item Une salle (pour éviter les murs et les obstacles fixes)
\end{itemize}
\vspace{3mm}

L'espace de travail est un rectangle de $6m\times4m$ dans lequel le robot peut évoluer sans risque de collision. Cet espace est couvert par les caméras de la \emph{motion capture}, il est donc possible de déterminer les positions de tous les éléments cités précédemment. 

D'autre part, le robot doit pouvoir être maintenu en cas de chute. Pour cela, un pont roulant 3 axes est mis en place au dessus de l'espace de travail. En attendant qu'il soit auto-piloté sur la position du robot, un manipulateur doit veiller à ce que les cordes de maintien soient toujours au dessus du robot. Sans cette protection, le robot ne serait plus maintenu en cas d'accident, et les dégats pourraient être considérables lors de l'utilisation de l'arrêt d'urgence.

\subsection{Cahier des charges}
\label{sub:cdc}

Les expériences à réaliser sont réparties en quatre objectifs :
\begin{itemize}
\item Planification avec évitement et franchissement d'obstacles :\\
\emph{Le robot être être capable de trouver un chemin d'un point A à un point B en évitant ou franchissant tous les obstacles.}
\item Replanification avec les obstacles mobiles :\\
\emph{Les obstacles pourront bouger dans l'espace de travail, ils ne devront pas être placés dans les 3 prochains pas du robot, ni au dessus de la position d'arrivée du robot.}
\item Suivi de personne :\\
\emph{Le robot doit atteindre une personne portant un casque suivi par la \emph{motion capture}, il doit s'arrêter dans un carré de $1m$ autour de la position de la personne.}
\item Suivi de personne en évitant les obstacles mobiles :\\
\emph{Le robot doit être capable d'effectuer l'évitement et le suivi en même temps.}
\end{itemize}



\subsection{Résultats}
\label{sub:resultats}

L'utilisation de PQP et l'optimisation du code-source du \emph{planner} ont permis d'augmenter d'environ 30 fois la vitesse de planification. Pour un même environnement complexe, contenant des obstacles 2D et trois cubes en 3D, le temps de planification moyen est passé de $29.8s$ à $0.98s$. Ceci a permis de confirmer que l'on pouvait utiliser notre algorithme pour de la replanification en temps réel.


\begin{figure}[h]
\begin{center}
\includegraphics[width=11.0cm]{images/SweptDemo.png}
\caption{Evitement de collisions avec des obstacles complexes (>10.000 triangles). Ici, seuls les volumes balayés des jambes ont été utilisés.}
\label{fig:demo}
\end{center}
\end{figure}

Les capacités des robots humanoïdes ont bien été exploitées comme le montre la figure~\ref{fig:over} où le robot enjambe une barre. Le robot doit pouvoir naviguer dans un environnement humain, l'objectif étant de prendre des cas comme une chambre d'enfant où de nombreux objets sont disposés aléatoirement sur le sol.

\begin{figure}[h]
\begin{center}
\includegraphics[width=13.0cm]{images/over_small.png}
\caption{Enjambement d'un obstacle de $8cm$ de haut lors de l'exécution d'une trajectoire planifiée par HRP-2.}
\label{fig:over}
\end{center}
\end{figure}

\newpage

\newpage

\section{Avancement}

\subsection{Bibliothèques testées}

\subsubsection{Reconstruction 3D}

La reconstruction 3D étant un domaine de recherche répandu, un grand nombre d'algorithmes existe déjà dans la théorie. J'ai donc tenté de trouver une librairie gratuite permettant de reconstruire un nuage de points à partir d'une liste d'images.
On peut trouver plusieurs algorithmes sur OpenSLAM.org :
\begin{itemize}
  \item ScaViSLAM
  \item RobotVision
\end{itemize}

Cependant les travaux présents sur OpenSLAM.org sont généralement conçus pour des capteurs de type laser. Ils nécessitent donc un temps de conversion afin de pouvoir les utiliser avec des capteurs de vision, ceci n'étant pas toujours réalisable.

Il faut également que je regarde ce que propose la librairie MRPT\footnote{Mobile Robot Programming Toolkit} qui est disponible dans les dépôts officiels d'Ubuntu et qui propose une partie SLAM.

\subsubsection{Fusion de cartes}

Afin de fusionner des cartes j'ai testé l'algorithme ICP sur deux nuages de points ayant un nombre de points plus ou moins important en commun.

%\begin{table}[h]
%  \begin{center}
%    \begin{tabular}{|c|c|c|c|c|c|}
%    \hline
%    Nb1 & Nb2 & commun & ratio & scale & succes \\
%    \hline
%    1200 & 1500 & 800 & 60\% & 1.0 & OK \\
%    1200 & 1500 & 800 & 60\% & 0.6 & Error \\
%    \hline
%    \end{tabular}
%  \caption{Réussite de l'ICP}
%  \end{center}
%\end{table}
Les résultats des tests réalisés montrent que pour pouvoir trouver la bonne solution, l'algorithme ICP a besoin d'un nombre assez important de points en commun entre les deux nuages de points.
Un pourcentage de points communs inférieur à 70\% environ entraîne un mauvais recalage des nuages de points. 
D'autre part, cet algorithme n'aboutit pas aux bons résultats si l'échelle est différentes entre les deux nuages.
Dans le cas d'un capteur laser, l'échelle est toujours la même, mais pour une caméra, l'échelle dépend principalement de la distance entre les deux premières vues.
L'algorithme ICP ne répond donc pas au cahier des charges imposé par l'utilisation de caméras.
Afin de faciliter la réalisation de la fusion de cartes sur un grand environnement, il faudrait que je trouve un autre algorithme permettant d'avoir un a priori sur la position relative des nuages.

\subsection{Travaux réalisés}

Plusieurs programmes ont été mis au point pendant le début de ma thèse.
Ils ont pour but de tester point par point les différentes fonctions à réalisées.
\begin{itemize}
\item Calibration : Effectue la calibration automatique d'une caméra perspective à partir d'une séquence d'images de mire
\item Création d'images :  Permet de simuler un environnement et de créer des images perspectives ou omnidirectionnelles ainsi que le nuage de points associé
\item Sélecteur de points : Affiche deux images côte à côte afin de choisir les points en correspondance manuellement par simple clique
\item Modificateur de points : Permet de déplacer et/ou de redimensionner un nuage de points 3D
\item Création 3D : Permet de créer un modèle 3D depuis des images ou depuis une vidéo, seule la version pour caméra perspective est opérationnelle à ce jour
\item ICP : Fonctions de base permettant de tester l'algorithme ICP
\item Bibliothèque de reconstruction 3D : Ensemble de fonctions utiles pour la reconstruction 3D ainsi que des outils pour travailler avec les images, les fichiers de points, ainsi que les fonctions mathématiques nécessaire pour résoudre les équations présentées dans la section~\ref{sec:vision}.
\end{itemize}

\vspace{5mm}
Les images~\ref{fig:create} et~\ref{fig:build} permettent de voir le résultat d'une simulation de reconstruction 3D d'un environnement à l'aide d'une caméra perspective.
\begin{figure}[htp]
\begin{center}
\subfloat[Nuage simulé]{\label{fig:create}\includegraphics[width=0.45\linewidth]{images/createimage.png}}
\subfloat[Nuage recréé]{\label{fig:build}\includegraphics[width=0.45\linewidth]{images/buildfromvideo.png}}
\caption{L'image~\ref{fig:create} représente le nuage de points simulé ainsi que les différentes prises de vues, l'axe de la caméra ($\vec{z}$) est en bleu. Dans l'image~\ref{fig:build} on peut voir le résultat obtenu lors de la création du nuage à partir des images précédemment simulées.}
\end{center}
\end{figure}
\subsection{Travaux restants}

\subsubsection{Points importants}

Un certain nombre de points importants reste cependant à explorer :
\begin{enumerate}
\item Vision : Afin de détecter les points en correspondance entre des images de types différents de manière automatique, nous avons besoin de tester d'autres descripteurs que SIFT $\rightarrow$ SURF, MI, MSER, Ogre, \dots
\item Ajustement de faisceaux : Dans le cas d'images provenant de caméras différentes et en prenant également en compte la dérive du facteur d'échelle
\item Temps réel : Établir une communication en temps réel entre les robots pour la fusion de carte (voir les travaux de R. \textsc{Aragues}~\cite{Aragues11PhD})
\item SoViN : Implémenter les fonctions manquantes dans SoViN puis la lier à ma bibliothèque 
\end{enumerate}

Ce projet étant étalé sur trois années, il reste donc deux ans et demi pour aborder en temps voulu les points précédents, je dois notamment mettre en place les idées suivantes :
\begin{enumerate}
\item Une surveillance dans un environnement dynamique et inconnu
\item Améliorer la localisation à l'aide de cartes topologiques ainsi qu'en utilisant les informations de position relatives entre les robots
\item Générer le suivi d'une cible à l'aide de plusieurs robots
\item Générer des déplacements en formation avec ou sans \emph{leader}
\end{enumerate}

%\subsubsection{Grant}
%Mettre un Grant ici \dag
\newpage

\section{Conclusion}

Malgré 15000 lignes de code pour le début d'une bibliothèque écrite en C++ pour réaliser mes travaux, je n'ai pas encore réussi à obtenir de résultats concluants pour la création et fusion de cartes pour la robotique.

Cette bibliothèque permet actuellement de construire un nuage de points 3D à partir d'une vidéo issue d'une caméra perspective calibrée.
Le cas de la caméra catadioptrique est presque opérationnel, la convergence du recalage 3D/2D n'est pas garantie et rend donc des résultats qui peuvent être aléatoires.

Le projet s'étend sur trois années et devra au final permettre de controler une flotte de robots de façon autonome.
Les six premiers mois de cette thèse, représentant mon stage de fin d'études, ont principalement été consacrés à l'étude bibliographique du domaine concerné par la thèse : la robotique mobile autonome, la vision omnidirectionnelle et la fusion de carte.
En parallèle à cette étude, une partie développement d'outils de base à été réalisé comme mentionné précédement.

J'ai développé ces outils à partir des cours théoriques que j'ai eu en master \emph{Image et Vision} mais également à partir de livre comme 
\citetitle{Hartley03Book} 
de 
\citeauthor{Hartley03Book} 
\cite{Hartley03Book} 
et d'articles comme
\citetitle{Puig08} 
de 
\citeauthor{Puig08} 
\cite{Puig08}.
J'aurais aimé pouvoir utiliser des bibliothèques ou programmes existants développés au sein du laboratoire mais je n'y ai pas eu accès (bibliothèques non libre). De plus une grande partie des chercheurs utilise des outils de prototypage tel Matlab, et donc peu de code performant en C++ sont partagés. Donc même si les méthodes sont connues et utilisées dans le laboratoire, j'ai du les redévelopper moi même afin d'avoir une bibliothèque adéquate, partagée avec la communauté de façon libre et gratuite.

\newpage

%Bibliographie
\section{Bibliographie}
%\bibliographystyle{plain}
%\bibliography{./papers}
\printbibliography
\newpage

%Annexes
%\appendix

\section{Publication Humanoids 2011}

Publication pour la conférence Humanoids 2011.\\
\emph{Real-time Replanning Using 3D Environment for Humanoid Robot}\\
\leo, \nicolas, \olivier, \thomas, \eiichi et \florent.




\includepdf[page=1]{humanoids11-lbaudouin.pdf} 
\includepdf[page=2]{humanoids11-lbaudouin.pdf} 
\includepdf[page=3]{humanoids11-lbaudouin.pdf} 
\includepdf[page=4]{humanoids11-lbaudouin.pdf} 
\includepdf[page=5]{humanoids11-lbaudouin.pdf} 
\includepdf[page=6]{humanoids11-lbaudouin.pdf} 


\section{Publication Transactions of Robotics 2011}

Publication pour Transactions of Robotics 2011.\\
\emph{Fast humanoid robot collision-free footstep planning
using swept volume approximations}\\
\nicolas, \olivier, \leo, \florent et \eiichi.

\includepdf[page=1]{troPaper.pdf}
\includepdf[page=2]{troPaper.pdf}
\includepdf[page=3]{troPaper.pdf}
\includepdf[page=4]{troPaper.pdf}
\includepdf[page=5]{troPaper.pdf}
\includepdf[page=6]{troPaper.pdf}
\includepdf[page=7]{troPaper.pdf}
\includepdf[page=8]{troPaper.pdf}
\includepdf[page=9]{troPaper.pdf}
\includepdf[page=10]{troPaper.pdf}
\includepdf[page=11]{troPaper.pdf}
\includepdf[page=12]{troPaper.pdf}

\end{document}
