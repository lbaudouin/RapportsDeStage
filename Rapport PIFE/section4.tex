\section{Avancement}

\subsection{Bibliothèques testées}

\subsubsection{Reconstruction 3D}

La reconstruction 3D étant un domaine de recherche répandu, un grand nombre d'algorithmes existe déjà. J'ai donc tenté de trouver une librairie permettant de reconstruire un nuage de points à partir d'une liste d'images.
On peut trouver plusieurs algorithmes sur OpenSLAM.org :
\begin{itemize}
  \item ScaViSLAM
  \item RobotVision
\end{itemize}

Cependant les travaux présents sur OpenSLAM.org sont généralement conçus pour des capteurs de type laser.

\subsubsection{Fusion de cartes}

Afin de fusionner des cartes j'ai testé l'algorithme ICP sur deux nuages de points ayant un nombre de points plus ou moins important en commun.

\begin{table}[h]
  \begin{center}
    \begin{tabular}{|c|c|c|c|c|c|}
    \hline
    Nb1 & Nb2 & commun & ratio & scale & succes \\
    \hline
    1200 & 1500 & 800 & 60\% & 1.0 & OK \\
    1200 & 1500 & 800 & 60\% & 0.6 & Error \\
    \hline
    \end{tabular}
  \caption{Réussite de l'ICP}
  \end{center}
\end{table}


\subsection{Travaux restants}

\subsubsection{Points importants}
\begin{enumerate}
\item Vision : tester SIFT, SURF, MI et autres descripteurs
\item Bundle adjustment : large scale dans le cas multi-type
\item Temps réel : communication pour la fusion online (voir Aragues)
\item Sovin : connecté SOVIN à ma librairie
\end{enumerate}

\subsubsection{Grant}

Mettre un Grant ici \dag