\section{Vision}

Cette partie sera une consacrée à la partie théorique de la reconstruction 3D.
Dans un premier temps nous verrons le cas de la reconstruction à partir d'une paire de caméras classiques.
Nous verrons ensuite le modèle unifié pour les caméras catadioptrique.
Nous aborderons au final la vision hybride.

\subsection{Reconstruction 3D}
\label{sub:reconstruction}

La reconstruction 3D à partir de deux vues perspectives a été très bien formalisé dans le livre de \citeauthor{Hartley03Book} \cite{Hartley03Book}.
En prenant deux vues d'une même caméra perspective, nous devons calculer le déplacement entre les lieux de prise de vue.
Celui ci est déterminé par une matrice de projection $\mathbf{P}$ qui se compose de deux matrices $\mathbf{R}$ et $\mathbf{t}$, respectivement la matrice de rotation et la matrice de translation.

La projection d'un point $\mathbf{X}$ dans l'espace donne le point $\mathbf{m}$ dans le plan image :
\begin{equation}
s \underbrace{\begin{pmatrix}u \\ v \\ 1\end{pmatrix}}_{\mathbf{m}} = 
\underbrace{\begin{pmatrix}f_u && 0 && u_0 \\ 0 && f_v && v_0 \\ 0 && 0 && 1\end{pmatrix}}_{\mathbf{K}} . 
\underbrace{\begin{pmatrix}r11 && r12 && r13 && t1 \\ r21 && r22 && r23 && t2 \\ r31 && r32 && r33 && t3\end{pmatrix}}_{[\mathbf{R~t}]} . 
\underbrace{\begin{pmatrix}X \\ Y \\ Z \\ 1\end{pmatrix}}_{\mathbf{X}} 
\end{equation}

\begin{equation}
s.\mathbf{m} = \mathbf{K}.\mathbf{P}.\mathbf{X}
\end{equation}
Avec : 
\begin{itemize}
\item $\mathbf{m}$ le point dans l'image
\item $\mathbf{P}$ la matrice de projection 
$\mathbf{P} = [ \mathbf{R} ~ \mathbf{t} ]$
\item $\mathbf{K}$ la matrice des paramètres intrinsèques (calibration)
\item $s$ est un facteur réel quelconque ($s \in \Re$)
\end{itemize}

\subsubsection{Calibration}
\label{subsub:calibration}
La calibration d'une caméra perspective est réalisé à partir d'une mire \cite{??}.

\subsubsection{Méthode}

Comme nous avons pu voir précédemment, les seuls informations que l'on va utiliser dans les images sont des points.
Pour donner un sens à ces listes de points, ils seront appairés.
C'est à dire que chaque point dans la première image devra avoir un point correspondant dans la seconde.

On va pouvoir calculer un matrice regroupant toutes les informations possible de tiré de cette configuration.
Cette matrice ce nommera la matrice fondamentale, notée $\mathbf{F}$.
Elle est définit pour toutes paires de points $i$ dans les images 1 et 2, par la suite l'indice $i$ sera omis.
\begin{equation}
\mathbf{m}_{i,1}.\mathbf{F}.\mathbf{m}_{i,2} = 0
\label{eq:fondamentale}
\end{equation}
Connaissant la matrice intrinsèque de la caméra (voir \ref{subsub:calibration}), nous pouvons éliminer les facteurs induit par les focales, afin de normaliser la matrice fondamentale.
On obtiendra la matrice essentielle $\mathbf{E}$ :
\begin{equation}
\mathbf{E} = \mathbf{K}_1^{T} . \mathbf{F} . \mathbf{K}_2
\end{equation}
Si les deux images sont acquis par la même caméra, on aura $\mathbf{K}_1 = \mathbf{K}_2$.

Nous pouvons obtenir la matrice essentielle directement depuis la liste de points, en normalisant les points eux-même.
\begin{equation}
(\mathbf{m}_{i,1}.\mathbf{K}_1^{-T}).\mathbf{E}.(\mathbf{K}_2^{-1}.\mathbf{m}_{i,2}) = 0
\label{eq:essentielle}
\end{equation}


Afin d'obtenir les matrices $\mathbf{R}$ et $\mathbf{t}$, nous devons utiliser une autre formulation de $\mathbf{E}$ :
\begin{equation}
\mathbf{E} = \mathbf{R} . [\mathbf{t}]_x
\end{equation}
Avec $[\mathbf{t}]_x$ la matrice anti-symétrique\footnote{Si $\mathbf{t}=\begin{pmatrix}t_1\\t_2\\t_3\end{pmatrix}$, alors $[\mathbf{t}]_x=\begin{pmatrix}0&-t_3&t_2\\t_3&0&-t_1\\-t_2&t_1&0\end{pmatrix}$} du vecteur $\mathbf{t}$.

Nous pouvons maintenant extraire $\mathbf{R}$ et $\mathbf{t}$. Pour celà nous devons décomposer la matrice essentielle :
$$\mathbf{E}=\mathbf{U} \mathbf{\Sigma} \mathbf{V}^T$$
Nous avons, $ \mathbf{\Sigma} = \begin{pmatrix}s&0&0\\0&s&0\\0&0&0\end{pmatrix}$, alors:
\begin{equation}
[\mathbf{t}]_x = \mathbf{V} \mathbf{W} \mathbf{\Sigma} \mathbf{V}^T
\end{equation}
\begin{equation}
\mathbf{R} = \mathbf{U} \mathbf{W}^{-1} \mathbf{V}^T
\end{equation}
Avec, $\mathbf{W}=\begin{pmatrix}0&1&0\\-1&0&0\\0&0&1\end{pmatrix}$




\subsection{Modèle unifié}

Pour les caméras catadioptriques, un modèle permettant de représenter les caméeas munies d'un mirroir plan, parabolique, hyperbolique, elliptique a été mis en place.

Ce modèle unifié, appelé modèle sphérique, prends en entré un seul paramètre $\xi$ pour représenter le mirroir.

\fig{images/videH}{Modèle sphérique pour caméras catadioptriques}{fig:modeleunifie}

\subsubsection{Modèle de projection}

Soit $\mathbf{X}$ un point dans le repère monde.
On suppose la caméra au point $O$ de coordonnées $(0,0,0)$ avec l'axe $\vec{z}$, l'axe principal de la caméra.

On va projeter le point $X$ sur la sphère unitaire centrer en $O$. On obtiendra le point $X_m$.
\begin{equation}
X_m = \frac{X}{\rho}
\end{equation}
Avec $\rho = ||X|| = \sqrt{X^2+Y^2+Z^2}$.

On projete enfin le point $X_m$ sur le plan de $Z = 1 - \xi$ :
\begin{equation}
x = \begin{bmatrix}  \frac{X}{Z+\xi \rho} && \frac{Y}{Z +\xi \rho} && 1 \end{bmatrix}^{T}
\end{equation}

Pour finir on applique la transformée homographique due aux paramètres intrinsèques de la caméra, pour obtenir le point $m$ dans l'image.
\begin{equation}
m= K x
\end{equation}

\subsubsection{Retroprojection}

A partir du point $m$ dans l'image on souhaite avoir le point 3D dans le repère monde correspondant.

\begin{equation}
x = \begin{bmatrix} x && y && 1 \end{bmatrix}^{T} = K^{-1} m
\end{equation}

On inverse la fonction de projection pour obtenir les coordonnées du point du la shpère :
\begin{equation}
X_m = (\nu^{-1} + \xi) \bar{x}
\end{equation}
\begin{equation}
\bar{x} = \begin{bmatrix}x && y && \frac{1}{1+\xi \mu} \end{bmatrix}^{T}
\end{equation}
Avec :
\begin{equation}
  \left \{
  \begin{matrix}
    \mu = \frac{-\gamma-\xi(x^2+y^2)}{\xi(x^2+y^2)-1} \\
    \gamma = \sqrt{1+(1-\xi^2)(x^2+y^2)}
  \end{matrix}
 \right.
\end{equation}



\subsection{Vision Hybride}
Puig \cite{Puig08}

Pour les points de l'image catadioptrique, nous devons utilisé les coordonnées \cite{lifted}:
$$\hat{\mathbf{q}}=\begin{pmatrix}q_1^2+q_2^2\\q_1q_2\\q_2q_3\\q_3^2\end{pmatrix}$$
Nous avons :
\begin{equation}
\hat{\mathbf{q}}_c^T\mathbf{F}_{cp}\mathbf{q}_p=0
\end{equation}
On défini $\mathbf{B}_c$ ($3\times4$ matrice) pour représenter la matrice de projection pour la caméra catadioptrique.
La matrice essentielle ne peut pas \^etre calculé comme dans la section~\ref{sub:reconstruction} car $\mathbf{B}_c$ n'est pas une matrice de rotation :
\begin{equation}
\mathbf{E} \neq \mathbf{B}_c^T \mathbf{F}_{cp} \mathbf{K}_p
\end{equation}
