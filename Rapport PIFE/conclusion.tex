\section{Conclusion}

Malgré plus de 15000 lignes de code pour le début d'une bibliothèque écrite en C++ afin réaliser mes travaux, je n'ai pas encore réussi à obtenir de résultats concluants pour la création et fusion de cartes pour la robotique.

Cette bibliothèque permet actuellement de construire un nuage de points 3D à partir d'une vidéo issue d'une caméra perspective calibrée.
Le cas de la caméra catadioptrique n'est pas encore complètement opérationnel.
La convergence du recalage entre les points 3D et les points 2D de l'image, n'est pas garantie et rend donc des résultats qui peuvent être aléatoires.

Le projet s'étend sur trois années et devra au final permettre de contrôler une flotte de robots de façon autonome.
Les six premiers mois de cette thèse, représentant mon stage de fin d'études, ont principalement été consacrés à l'étude bibliographique du domaine concerné par la thèse : la robotique mobile autonome, la vision omnidirectionnelle et la fusion de carte.
En parallèle à cette étude, une partie développement d'outils de base à été réalisé comme mentionné précédemment.

J'ai développé ces outils à partir des cours théoriques que j'ai eu en master \emph{Image et Vision} mais également à partir de livre comme 
\citetitle{Hartley03Book} 
de 
\citeauthor{Hartley03Book} 
\cite{Hartley03Book} 
et d'articles comme
\citetitle{Puig08} 
de 
\citeauthor{Puig08} 
\cite{Puig08}.

J'aurais aimé pouvoir utiliser des bibliothèques ou programmes existants développés au sein du laboratoire ou dans d'autres laboratoires similaires mais je n'y ai pas encore eu accès.
De plus une grande partie des chercheurs utilisent des outils de prototypage tel Matlab, et donc peu de code performant en C++ sont partagés.
Donc même si les méthodes sont connues et utilisées dans le laboratoire, j'ai du les re-développer moi même afin d'avoir une bibliothèque adéquate, partagée avec la communauté de façon libre et gratuite.

Concernant la vitesse d'avancement du projet, je considère que beaucoup de temps a été utilisé pour l'étude bibliographique.
Une autre partie à pour l'instant été consacrée à la programmation d'outils afin d'acquérir les connaissances nécessaires pour la reconstruction 3D.
Il est donc normal que peu de résultats soient encore exploitables.
